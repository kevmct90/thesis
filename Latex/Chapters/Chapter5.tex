% Chapter Template

\chapter{Experiment Implementation \& Evaluation} % Main chapter title

\label{Chapter5} % Change X to a consecutive number; for referencing this chapter elsewhere, use \ref{ChapterX}

\lhead{Chapter 5. \emph{Experiment Implementation \& Evaluation}} % Change X to a consecutive number; this is for the header on each page - perhaps a shortened title

%----------------------------------------------------------------------------------------
%	SECTION 1
%----------------------------------------------------------------------------------------
\section{Introduction}
This chapter describes the experiment set-up and methodology for this project to predict if SME customers will go into arrears. The main theory being examined is that macro location based features/metrics can be indicative of SME's in the future going into arrears.  

There is relatively limited literature to support the location-based features chosen in this thesis, decisions are largely made due to anecdotal evidence and subject matter experts knowledge. To address this issue, empirical analysis of these features will be completed where one set of results will cascade new results and inspire further analysis. 

To understand if the location-based features in this analysis are useful, an initial model will be used as a baseline model from which we will look to improve iteratively. The features from this model will be derived from a historic scorecard used in \subjectname.  

\section{Conclusion}