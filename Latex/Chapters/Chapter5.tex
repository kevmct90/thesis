% Chapter Template

\chapter{Implementation \& Evaluation} % Main chapter title

\label{Chapter5} % Change X to a consecutive number; for referencing this chapter elsewhere, use \ref{ChapterX}

\lhead{Chapter 5. \emph{Experiment Implementation \& Evaluation}} % Change X to a consecutive number; this is for the header on each page - perhaps a shortened title

%----------------------------------------------------------------------------------------
%	SECTION 1
%----------------------------------------------------------------------------------------
\section{Introduction}
This chapter will present the implementation of the experiments to evaluate the use of macro-economic features for predicting SME defaults. The predictive capability of customers spending behaviour, personal customer default rates, SME default rates and Census data  will be analysed and discussed.

The experiments in this research are sequential meaning results from one experiment are used in the following experiment. Therefore results will also be discussed and evaluated in this section.

A benchmark predictive model will be built in SAS and R based on historical scorecard features. This will be used to make fair comparisons to the results from the experiments.


\section{Benchmark Models}
As mentioned in previous section when building predictive models its is essential to have a baseline or benchmark model to compare your experiments to. Therefore benchmark regression models have been built in R and SAS that will be used to make comparisons to experiments in this chapter. Regression was chosen because of its success in experiments in Section \ref{sec:benchFeature} and its wide use in industry. The models have been trained for both R and SAS using a 70\% stratified sample dataset with 30\% being kept for holdout which will be used for testing. Results may vary from both R and SAS due to varying samples in each application and algorithms will be slightly different but they should relatively close. This will not impact analysis of comparing experiments as experiments in R will be compared to the benchmark from R and likewise for comparisons in SAS. The AUC will the performance measure of choice for evaluating the well the model fits over all possible thresholds. As the AUC does not give us the accuracy of the model a specific threshold we will also use the recall, specificity and balanced accuracy(BA). The threshold used is based on fine tuning of the parameters testing using a validation dataset in Section \ref{sec:benchFeature}. The breakdown of the training and test for the benchmark model datasets and their target class distribution can be shown in Table \ref{table:benchmark_holdout_train_test}.

\begin{table}[H]
	\centering\
	\resizebox{\textwidth}{!}
	{
		\begin{tabular}{l l r r r r}
			\hline
			\textbf{Model} &  \textbf{Dataset} & \textbf{\# Bad} & \textbf{\# Good} & \textbf{\# Observations} & \textbf{Good:Bad} \\
			\hline
			Previous Delinquency & Training       & 483 & 1,565 & 2,048 & 76:24\\
			          & Test & 245 & 633 & 878 & 72:28\\\hline
			\textbf{Previous Previous Delinquency}     & \textbf{Total} & \textbf{728} & \textbf{2,198} & \textbf{2,926} & \textbf{75:25} \\
			\hline
			No Previous Delinquency & Training & 474 & 16,435 & 16,909 & 97:03 \\ 
			          & Test & 177 & 7,070 & 7,247 & 97:03 	\\\hline
			\textbf{No Previous Delinquency}     & \textbf{Total} & \textbf{651} & \textbf{23,505} & \textbf{24,156} & \textbf{97:03} \\
			\hline
			\textbf{Total } 	&     	     & \textbf{1,379} & \textbf{25,703} & \textbf{27,082} & \textbf{95:05}\\ \hline
		\end{tabular}
	}
	\caption{Breakdown Holdout Training/Test Dataset \\for Benchmark Models}
	\label{table:benchmark_holdout_train_test}
\end{table}

The threshold used for performance measuring is based on fine tuning of the parameters testing using a validation dataset in Section \ref{sec:benchFeature}. The results from the benchmark models can be found in Table \ref{table:benchmodel} below.

\begin{table}[H]
	\centering
	\resizebox{\textwidth}{!}
	{\footnotesize
		\begin{tabular}{l l l r r r r}
			\hline
			\textbf{Model} & \textbf{Dataset} & \textbf{Software} & \textbf{Recall} & \textbf{Specificity} & \textbf{BA} & \textbf{AUC}  \\ \hline
			\textit{PD\_Bench\_R} & Previous Delinquency & R & 0.54 & 0.70 & 0.62 & 0.654   \\ 
			\textit{PD\_Bench\_SAS} & Previous Delinquency & SAS & 0.53 & 0.62 & 0.57 & 0.62   \\ \hline
			\textit{NPD\_Bench\_R} & No Previous Delinquency & R & 0.53 & 0.74 & 0.63 & 0.65   \\ 
			\textit{NPD\_Bench\_SAS} & No Previous Delinquency & SAS & 0.49 & 0.74 & 0.62 & 0.65   \\ \hline
		\end{tabular}
	}
	\caption{Benchmark Model Results for Experiment Comparison}
	\label{table:benchmodel}
\end{table}

Models in this this section will be aliased for ease of reading, for example the benchmark models will be referred to as \textit{PD\_Bench\_R}, \textit{PD\_Bench\_SAS}, \textit{NPD\_Bench\_R} and \textit{NPD\_Bench\_SAS} which can be seen in Table \ref{table:benchmodel} above. The results found here are consistent with the evaluations completed in Chapter \ref{Chapter4} which guarantees we have good benchmarks to compare the results in the experiment in this chapter to.


\section{Correlation Analysis}
Correlation matrix is a very simple and powerful way of analysing the relationship of predictive features with each other and the target features. Including highly correlated predictive features has the potential to throw off or fool your predictive model potentially causing misleading results. The Pearson correlation coefficient is a common measure used to test your data to check this relationship and will be deployed in this experiment. The correlation tests are ran separately on each category of features created as part of this experiment e.g. \textit{CSO Features by ED \& LA}, \textit{SME Default Trends by ED \& LA}, \textit{Homeloan and Personal Loan Default Trends by ED \& LA}, \textit{Transactions Behaviour by ED \& LA}, and \textit{Binned SME Defaults Rates by ED and LA}. Fig. \ref{fig:unbal_corr_analysis} shows a subset of the results demonstrating correlation scores between the experiment features categories with a full breakdown of the results available in Appendix \ref{AppendixB} where a correlation score coloured red represents a very strong positive correlation between features and blue a very low negative correlation.

\begin{figure}[H]
	\centering
 	\begin{subfigure}[b]{0.32\textwidth}
 		\captionsetup{font=scriptsize}
 		\includegraphics[width=\textwidth,height=2.75cm]{CSO_Correlation_Analysis}
 		\caption{CSO Features by ED \& LA\\}
 		\label{fig:CSOCorrelation}
 	\end{subfigure}
	\begin{subfigure}[b]{0.32\textwidth}
		\captionsetup{font=scriptsize}
		\includegraphics[width=\textwidth,height=2.75cm]{SME_Arrears_Trends_Correlation_Analysis_Subset}
		\caption{SME Default Trends \\by ED \& LA}\label{fig:smeArrearsCorrelation}
	\end{subfigure} 
	\begin{subfigure}[b]{0.32\textwidth}
		\captionsetup{font=scriptsize}
		\includegraphics[width=\textwidth,height=2.75cm]{Personal_Arrears_Ratio}
		\caption{Homeloan and Personal Loan Default Trends by ED \& LA}
		\label{fig:personalArrearsCorrelation}
	\end{subfigure} 
	\medskip
	\begin{subfigure}[b]{0.32\textwidth}
		\captionsetup{font=scriptsize}
		\includegraphics[width=\textwidth,height=2.75cm]{Visa_Debit_Correlation_Analysis_Subset}
		\caption{Transactions Behaviour \\by ED \& LA }\label{fig:transVisaCorrelation}
	\end{subfigure} ~\quad
	\begin{subfigure}[b]{0.32\textwidth}
		\captionsetup{font=scriptsize}
		\includegraphics[width=\textwidth,height=2.75cm]{Grouped_Features_Correlation_Analysis_Subset}
		\caption{Binned SME Defaults Rates \\by ED and LA\\}
		\label{fig:groupedFeaturesCorrelation}
	\end{subfigure}
	\caption{Correlation Analysis}
	\label{fig:unbal_corr_analysis}
\end{figure}

There is a high level of correlation between among many of the variables. It was decided that any features that had a correlation score of 0.80 would be candidate features for removal. For features that were this highly correlated the feature scoring the lowest bivariate correlation to the target feature was removed, ensuring the variable which had the strongest relationship with the target class was prioritised and kept for further predictions. The a subset of the results after this feature reduction process can be found in Fig. \ref{unbal_corr_analysis_filtered} with a full set of results available in Appendix \ref{AppendixC}. 

\begin{figure}[H]
	\centering
	\begin{subfigure}[b]{0.32\textwidth}
		\captionsetup{font=scriptsize}
		\includegraphics[width=\textwidth,height=2.75cm]{CSO_Correlation_Analysis}
		\caption{CSO Features by ED \& LA\\}
		\label{fig:CSOCorrelation}
	\end{subfigure}
	\begin{subfigure}[b]{0.32\textwidth}
		\captionsetup{font=scriptsize}
		\includegraphics[width=\textwidth,height=2.75cm]{SMETrendsCorrSub}
		\caption{SME Default Trends \\by ED \& LA}\label{fig:SMETrendsCorrSub}
	\end{subfigure} 
	\begin{subfigure}[b]{0.32\textwidth}
		\captionsetup{font=scriptsize}
		\includegraphics[width=\textwidth,height=2.75cm]{PersonalTrendFull}
		\caption{Homeloan and Personal Loan Default Trends by ED \& LA}
		\label{fig:PersonalTrendFull}
	\end{subfigure} 
	\medskip
	\begin{subfigure}[b]{0.32\textwidth}
		\captionsetup{font=scriptsize}
		\includegraphics[width=\textwidth,height=2.75cm]{TransactionCorrSub}
		\caption{Transactions Behaviour \\by ED \& LA }\label{fig:TransactionCorrSub}
	\end{subfigure} ~\quad
	\begin{subfigure}[b]{0.32\textwidth}
		\captionsetup{font=scriptsize}
		\includegraphics[width=\textwidth,height=2.75cm]{GroupedCorrSub}
		\caption{Binned SME Defaults Rates \\by ED and LA\\}
		\label{fig:GroupedCorrSub}
	\end{subfigure}
	\caption{Correlation Analysis after highly correlated features have been removed}
	\label{fig:unbal_corr_analysis_filtered}
\end{figure}

We then analyse the variables that are most correlated with the target to see if included in the prediction model how well it performs, results from this analysis are demonstrated in Fig. \ref{fig:Correlation Analysis} below.

\begin{figure}[H]
	\includegraphics[width=0.9\textwidth,center]{CorrelationChart}
	\caption{Most correlated features with target variable}
	\label{fig:Correlation Analysis}
\end{figure}

Table \ref{table:CorrModelResults} below shows the results from including the top 5, 10, 15, 20 correlated features after the correlation analysis as per the Fig. \ref{fig:Correlation Analysis}. 

\begin{table}[H]
\centering
\small
		\begin{tabular}{l r r r r}
			\hline
			\textbf{Model} & \textbf{Recall} & \textbf{Specificity} & \textbf{BA} & \textbf{AUC}  \\ \hline
			\textit{PD\_Cor5\_R}  & 0.534 & 0.685 & 0.610 & 0.654   \\ 
			\textit{PD\_Cor10\_R} & 0.556 & 0.679 & 0.618 & 0.658  \\ 
			\textit{PD\_Cor15\_R} & 0.594 & 0.690 & 0.643 & 0.665  \\
			\textit{PD\_Cor20\_R} & 0.561 & 0.698 & 0.630 & 0.665  \\\hline 
			
			\textit{NPD\_Cor5\_R}  & 0.525 & 0.743 & 0.634 & 0.671   \\ 
			\textit{NPD\_Cor10\_R} & 0.555 & 0.725 & 0.640 & 0.677  \\ 
			\textit{NPD\_Cor15\_R} & 0.538 & 0.723 & 0.630 & 0.680  \\
			\textit{NPD\_Cor20\_R} & 0.549 & 0.731 & 0.640 & 0.676  \\\hline 
		\end{tabular}
	\caption{Models from Correlation Feature Selection}
	\label{table:CorrModelResults}
\end{table}

For each model result the AUC will be used to measure the overall accuracy of the model while Recall, Specificity, Balanced Accuracy will be recorded to evaluate how the model performs at specified threshold. These results will be evaluated against the benchmark results in a later section of the chapter.


\section{Information Gain \& Random Forest Feature Selection}

\footnote{\url{https://cran.r-project.org/web/packages/FSelector/index.html}}

\subsubsection{Information Gain Delinquency Model Existing \& Experimental Features}

\begin{figure}[H]
	\includegraphics[width=0.9\textwidth,center]{DelinqInformationGainAnalysis}
	\caption{Macro-Economic Feature Importance using Information Gain}
	\label{fig:DelinqInformationGainAnalysis}
\end{figure}


\begin{table}[H]
\centering
\small
		\begin{tabular}{l r r r r}
			\hline
			\textbf{Model} & \textbf{Recall} & \textbf{Specificity} & \textbf{BA} & \textbf{AUC}  \\ \hline
			\textit{PD\_IG5\_R} & 0.540 & 0.715 & 0.627 & 0.652   \\ 
			\textit{PD\_IG10\_R} & 0.548 & 0.693 & 0.621 & 0.649  \\ 
			\textit{PD\_IG15\_R} & 0.536 & 0.690 & 0.613 & 0.65  \\
			\textit{PD\_IG20\_R} & 0.544 & 0.711 & 0.628 & 0.651 \\\hline 
		\end{tabular}
	\caption{Models from Information Gain Feature Selection for Previous Delinquency}
	\label{table:InfoGainPDModelResults}
\end{table}


\subsubsection{Information Gain No Delinquency Model Existing \& Experimental Features}

\begin{figure}[H]
	\includegraphics[width=0.9\textwidth,center]{NoDelinqInformationGainAnalysis}
	\caption{Information Gain No Previous Delinquency Model Features}
	\label{fig:NoDelinqInformationGainAnalysis}
\end{figure}


\begin{table}[H]
\centering
\small
		\begin{tabular}{l r r r r}
			\hline
			\textbf{Model} & \textbf{Recall} & \textbf{Specificity} & \textbf{BA} & \textbf{AUC}  \\ \hline
			\textit{NPD\_IG5\_R} & 0.542 & 0.736 & 0.639 & 0.667   \\ 
			\textit{NPD\_IG10\_R} & 0.536 & 0.736 & 0.636 & 0.667  \\ 
			\textit{NPD\_IG15\_R} & 0.520 & 0.732 & 0.626 & 0.663 \\
			\textit{NPD\_IG20\_R} &  0.508 & 0.730 & 0.619 & 0.665  \\\hline 
		\end{tabular}
	\caption{Models from Information Gain Feature Selection for No Previous Delinquency}
	\label{table:InfoGainNPDModelResults}
\end{table}

\subsubsection{Random Forest Delinquency Model Existing \& Experimental Features}

\begin{figure}[H]
	\includegraphics[width=0.9\textwidth,center]{DelinqRandomForestsAnalysis}
	\caption{Macro-Economic Feature Importance using Random Forests}
	\label{fig:DelinqRandomForestsAnalysis}
\end{figure}

\begin{table}[H]
\centering
\small
		\begin{tabular}{l r r r r}
			\hline
			\textbf{Model} & \textbf{Recall} & \textbf{Specificity} & \textbf{BA} & \textbf{AUC}  \\ \hline
			\textit{PD\_RF5\_R} & 0.548 & 0.692 & 0.620 & 0.662   \\ 
			\textit{PD\_RF10\_R} & 0.556 & 0.703 & 0.630 & 0.655  \\ 
			\textit{PD\_RF15\_R} & 0.548 & 0.698 & 0.623 & 0.654  \\
			\textit{PD\_RF20\_R} & 0.535 & 0.698 & 0.617 & 0.651  \\\hline 
		\end{tabular}

	\caption{Models from Random Forest Feature Selection for Previous Delinquency}
	\label{table:RFPDModelResults}
\end{table}



\subsubsection{Random Forest No Delinquency Model Existing \& Experimental Features}

\begin{figure}[H]
	\includegraphics[width=0.9\textwidth,center]{NoDelinqRandomForestsAnalysis}
	\caption{Macro-Economic Feature Importance using Random Forests}
	\label{fig:NoDelinqRandomForestsAnalysis}
\end{figure}


\begin{table}[H]
\centering
\small
		\begin{tabular}{l  r r r r}
			\hline
			\textbf{Model}  & \textbf{Recall} & \textbf{Specificity} & \textbf{BA} & \textbf{AUC}  \\ \hline
			\textit{NPD\_RF5\_R}  & 0.534 & 0.685 & 0.610 & 0.654   \\ 
			\textit{NPD\_RF10\_R} & 0.556 & 0.679 & 0.618 & 0.658  \\ 
			\textit{NPD\_RF15\_R} & 0.594 & 0.690 & 0.643 & 0.665  \\
			\textit{NPD\_RF20\_R} & 0.561 & 0.698 & 0.630 & 0.665  \\\hline 
		\end{tabular}

	\caption{Models from Random Forest Feature Selection for No Previous Delinquency}
	\label{table:RFNPDModelResults}
\end{table}


\section{Interactive Grouping}

\subsubsection{Previous Delinquency}
\begin{figure}[H]
	\includegraphics[width=0.9\textwidth,center]{PreviousDelinq_InformationGain_InteractiveGrouping}
	\caption{Information Value using SAS Previous Delinquency Features}
	\label{fig:Information Value using SAS Previous Delinquency Features}
\end{figure}

\begin{figure}[H]
	\includegraphics[width=0.7\textwidth,center]{ExampleInteractiveGrouping}
	\caption{Interactive Grouping Diff Percent 06 2012 ED}
	\label{fig:Interactive Grouping Diff Percent 06 2012 ED}
\end{figure}

\begin{figure}[H]
	\includegraphics[width=0.7\textwidth,center]{PDCoarseSelectionVariableImportanceInformationGain}
	\caption{Coarse Selection Feature Importance Previous Delinquency by Information Gain and Gini Statistic}
	\label{fig:PDCoarseSelectionVariableImportanceInformationGain}
\end{figure}

\begin{table}[H]
	\centering
	\small
	\begin{tabular}{l r r r r}
		\hline
		\textbf{Model} & \textbf{Recall} & \textbf{Specificity} & \textbf{BA} & \textbf{AUC}  \\ \hline
		\textit{PD\_Coarse5\_SAS}  & 0.543 & 0.645 & 0.694 & 0.627   \\ 
		\textit{PD\_Coarse10\_SAS} & 0.552 & 0.618 & 0.585 & 0.619  \\ 
		\textit{PD\_Coarse15\_SAS} & 0.511 & 0.618 & 0.564 & 0.59  \\
		\textit{PD\_Coarse20\_SAS} & 0.511 & 0.625 & 0.568 & 0.59  \\\hline 
	\end{tabular}
	\caption{Previous Delinquency Models from Coarse Selection using Information Gain}
	\label{table:CoarsePDModelResults}
\end{table}


\subsubsection{No Previous Delinquency}

\begin{figure}[H]
	\includegraphics[width=0.9\textwidth,center]{NoPreviousDelinq_InformationGain_InteractiveGrouping}
	\caption{Information Value using SAS No Previous Delinquency Features}
	\label{fig:Information Value using SAS No Previous Delinquency Features}
\end{figure}

\begin{figure}[H]
	\includegraphics[width=0.7\textwidth,center]{ExampleInteractiveGroupingNPD}
	\caption{Interactive Grouping Event Rate Chart top 5}
	\label{fig:ExampleInteractiveGroupingNPD}
\end{figure}

\begin{figure}[H]
	\includegraphics[width=0.7\textwidth,center]{NPDCoarseSelectionVariableImportanceInformationGain}
	\caption{Coarse Selection Feature Importance No Previous Delinquency by Information Gain and Gini Statistic}
	\label{fig:NPDCoarseSelectionVariableImportanceInformationGain}
\end{figure}

\begin{table}[H]
	\centering
	\small
	\begin{tabular}{l r r r r}
		\hline
		\textbf{Model} & \textbf{Recall} & \textbf{Specificity} & \textbf{BA} & \textbf{AUC}  \\ \hline
		\textit{NPD\_Coarse5\_SAS}  & 0.492 & 0.738 & 0.615 & 0.664   \\ 
		\textit{NPD\_Coarse10\_SAS} & 0.497 & 0.742 & 0.619 & 0.667  \\ 
		\textit{NPD\_Coarse15\_SAS} & 0.517 & 0.741 & 0.629 & 0.669  \\
		\textit{NPD\_Coarse20\_SAS} & 0.50 & 0.739 & 0.619 & 0.665  \\\hline 
	\end{tabular}
	\caption{No Previous Delinquency Models from Coarse Selection using Information Gain}
	\label{table:CoarseNPDModelResults}
\end{table}

\section{Oversampling}

\begin{table}[H]
	\centering\
	\resizebox{\textwidth}{!}
	{
		\begin{tabular}{l l r r r r}
			\hline
			\textbf{Model} &  \textbf{Dataset} & \textbf{\# Bad} & \textbf{\# Good} & \textbf{\# Observations} & \textbf{Good:Bad} \\
			\hline
			Previous Delinquency & Training Oversample & 1,562 & 1,552 & 3,114 & 50:50\\
			& Test & 245 & 633 & 878 & 72:28\\\hline
			\textbf{Previous Previous Delinquency}     & \textbf{Total} & \textbf{1,807} & \textbf{2,185} & \textbf{3,992} & \textbf{55:45} \\
			\hline
			No Previous Delinquency & Training Oversample & 16,198 & 16,435 & 32,633 & 50:50 \\ 
			& Test & 177 & 7,070 & 7,247 & 97:03 	\\\hline
			\textbf{No Previous Delinquency}     & \textbf{Total} & \textbf{16,375} & \textbf{23,505} & \textbf{39,880} & \textbf{68:32} \\
			\hline
			\textbf{Total } 	&     	     & \textbf{18,182} & \textbf{25,690} & \textbf{43,872} & \textbf{58:42}\\ \hline
		\end{tabular}
	}
	\caption{Breakdown Holdout Training/Test Dataset \\for Oversample Benchmark Models}
	\label{table:benchmark_holdout_oversample_train_test}
\end{table}

\begin{table}[H]
	\centering
	\resizebox{\textwidth}{!}
	{
		\begin{tabular}{l l l r r r r}
			\hline
			\textbf{Model} & \textbf{Dataset} & \textbf{Software} & \textbf{Recall} & \textbf{Specificity} & \textbf{BA} & \textbf{AUC}  \\ \hline
			\textit{PD\_OverBench\_R} & Previous Delinquency & R & 0.587 & 0.615 & 0.601 & 0.651   \\ \hline
			\textit{NPD\_OverBench\_R} & No Previous Delinquency & R & 0.559 & 0.720 & 0.640 & 0.67   \\ \hline
		\end{tabular}
	}
	\caption{Benchmark Oversampling Model results for Comparison}
	\label{table:benchmodelOver}
\end{table}

\begin{table}[H]
\centering
\small
		\begin{tabular}{l  r r r r}
			\hline
			\textbf{Model} & \textbf{Recall} & \textbf{Specificity} & \textbf{BA} & \textbf{AUC}  \\ \hline
			\textit{PD\_OverExper\_R}  & 0.567 & 0.647 & 0.607 & 0.646   \\ \hline
			\textit{NPD\_OverExper\_R} & 0.587 & 0.711 & 0.649 & 0.671   \\ \hline
		\end{tabular}
	\caption{Experiment Oversampling Model Results for Experiment Comparison}
	\label{table:benchmodelOverExper}
\end{table}



\section{Synthetic Sampling}
NonDelinqknnFit

k-Nearest Neighbors 


16408 samples

116 predictor

2 classes: 'N', 'Y' 


Pre-processing: centered (116), scaled (116) 

Resampling: Cross-Validated (10 fold, repeated 3 times) 

Summary of sample sizes: 14768, 14767, 14766, 14767, 14768, 14767, ... 

Resampling results across tuning parameters:

\begin{figure}[H]
	\includegraphics[width=0.9\textwidth,center]{NoPreviousDelinquencyExperimentKNN}
	\caption{Repeated 3 Times 10-Fold Cross-Validation Optimal No. of Neighbours for No Previous Delinquency Experiment Data}
	\label{fig:NoPreviousDelinquencyExperimentKNN}
\end{figure}

{\footnotesize
	\begin{longtable}
		{l | l | l | l | l | l | l}
		\hline
		\textbf{k} & \textbf{ROC} & \textbf{Sens} & \textbf{Spec} & \textbf{ROCSD} & \textbf{SensSD} & \textbf{SpecSD} \\ \hline
		5          & 0.474        & 1.000         & 0.000         & 0.023          & 0.000           & 0.000           \\
		7          & 0.469        & 1.000         & 0.000         & 0.024          & 0.000           & 0.000           \\
		9          & 0.467        & 1.000         & 0.000         & 0.030          & 0.000           & 0.000           \\
		11         & 0.462        & 1.000         & 0.000         & 0.030          & 0.000           & 0.000           \\
		13         & 0.468        & 1.000         & 0.000         & 0.027          & 0.000           & 0.000           \\
		15         & 0.471        & 1.000         & 0.000         & 0.024          & 0.000           & 0.000           \\
		17         & 0.479        & 1.000         & 0.000         & 0.031          & 0.000           & 0.000           \\
		19         & 0.485        & 1.000         & 0.000         & 0.042          & 0.000           & 0.000           \\
		21         & 0.501        & 1.000         & 0.000         & 0.048          & 0.000           & 0.000           \\
		23         & 0.520        & 1.000         & 0.000         & 0.049          & 0.000           & 0.000           \\
		25         & 0.532        & 1.000         & 0.000         & 0.045          & 0.000           & 0.000           \\
		27         & 0.535        & 1.000         & 0.000         & 0.041          & 0.000           & 0.000           \\
		29         & 0.547        & 1.000         & 0.000         & 0.033          & 0.000           & 0.000           \\
		31         & 0.548        & 1.000         & 0.000         & 0.034          & 0.000           & 0.000           \\
		33         & 0.550        & 1.000         & 0.000         & 0.032          & 0.000           & 0.000           \\
		35         & 0.541        & 1.000         & 0.000         & 0.043          & 0.000           & 0.000           \\
		37         & 0.475        & 1.000         & 0.000         & 0.056          & 0.000           & 0.000           \\
		39         & 0.452        & 1.000         & 0.000         & 0.040          & 0.000           & 0.000           \\
		41         & 0.447        & 1.000         & 0.000         & 0.033          & 0.000           & 0.000           \\
		43         & 0.447        & 1.000         & 0.000         & 0.035          & 0.000           & 0.000           \\
		45         & 0.444        & 1.000         & 0.000         & 0.041          & 0.000           & 0.000           \\
		47         & 0.454        & 1.000         & 0.000         & 0.052          & 0.000           & 0.000           \\
		49         & 0.476        & 1.000         & 0.000         & 0.067          & 0.000           & 0.000           \\
		51         & 0.481        & 1.000         & 0.000         & 0.071          & 0.000           & 0.000           \\
		53         & 0.501        & 1.000         & 0.000         & 0.073          & 0.000           & 0.000           \\
		55         & 0.511        & 1.000         & 0.000         & 0.070          & 0.000           & 0.000           \\
		57         & 0.509        & 1.000         & 0.000         & 0.069          & 0.000           & 0.000           \\
		59         & 0.508        & 1.000         & 0.000         & 0.065          & 0.000           & 0.000           \\
		61         & 0.517        & 1.000         & 0.000         & 0.066          & 0.000           & 0.000           \\
		63         & 0.525        & 1.000         & 0.000         & 0.064          & 0.000           & 0.000           \\
		65         & 0.536        & 1.000         & 0.000         & 0.053          & 0.000           & 0.000           \\
		67         & 0.536        & 1.000         & 0.000         & 0.052          & 0.000           & 0.000           \\
		69         & 0.545        & 1.000         & 0.000         & 0.051          & 0.000           & 0.000           \\
		71         & 0.552        & 1.000         & 0.000         & 0.047          & 0.000           & 0.000           \\
		73         & 0.552        & 1.000         & 0.000         & 0.044          & 0.000           & 0.000           \\
		\cellcolor{green!25}75         &  \cellcolor{green!25}0.556        &  \cellcolor{green!25}1.000         &  \cellcolor{green!25}0.000         &  \cellcolor{green!25}0.038          &  \cellcolor{green!25}0.000           &  \cellcolor{green!25}0.000           \\
		77         & 0.541        & 1.000         & 0.000         & 0.052          & 0.000           & 0.000           \\
		79         & 0.527        & 1.000         & 0.000         & 0.061          & 0.000           & 0.000           \\
		81         & 0.476        & 1.000         & 0.000         & 0.064          & 0.000           & 0.000           \\
		83         & 0.456        & 1.000         & 0.000         & 0.056          & 0.000           & 0.000          \\ \hline
		\caption{No Previous Delinquency Model Analysis K-NN}
		\label{No Previous Delinquency Model Analysis K-NN}
	\end{longtable}
}

ROC was used to select the optimal model using  the largest value.

The final value used for the model was k = 75. 




DelinqknnFit

k-Nearest Neighbors 

1967 samples

118 predictor

2 classes: 'N', 'Y' 


Pre-processing: centered (118), scaled (118) 

Resampling: Cross-Validated (10 fold, repeated 3 times) 

Summary of sample sizes: 1770, 1771, 1770, 1771, 1770, 1770, ... 

Resampling results across tuning parameters:

\begin{figure}[H]
	\includegraphics[width=0.9\textwidth,center]{DelinqencyAnalysisKNN}
	\caption{Repeated 3 Times 10-Fold Cross-Validation Optimal No. of Neighbours for Previous Delinquency Experiment Data}
	\label{fig:DelinqencyAnalysisKNN}
\end{figure}

{\footnotesize
	\begin{longtable}
		{l|l|l|l|l|l|l}
		\hline
		\textbf{k} & \textbf{ROC} & \textbf{Sens} & \textbf{Spec} & \textbf{ROC SD} & \textbf{Sens SD} & \textbf{Spec SD} \\ \hline
		5          & 0.486        & 0.886         & 0.130         & 0.053           & 0.030            & 0.043            \\
		7          & 0.507        & 0.920         & 0.101         & 0.060           & 0.023            & 0.049            \\
		9          & 0.509        & 0.937         & 0.086         & 0.057           & 0.019            & 0.032            \\
		11         & 0.514        & 0.952         & 0.059         & 0.054           & 0.015            & 0.030            \\
		13         & 0.517        & 0.961         & 0.049         & 0.052           & 0.019            & 0.024            \\
		15         & 0.525        & 0.967         & 0.034         & 0.048           & 0.018            & 0.024            \\
		17         & 0.521        & 0.976         & 0.020         & 0.054           & 0.014            & 0.017            \\
		19         & 0.533        & 0.981         & 0.020         & 0.046           & 0.013            & 0.019            \\
		21         & 0.525        & 0.988         & 0.016         & 0.050           & 0.009            & 0.014            \\
		23         & 0.529        & 0.993         & 0.008         & 0.048           & 0.006            & 0.010            \\
		25         & 0.516        & 0.997         & 0.006         & 0.052           & 0.006            & 0.011            \\
		27         & 0.528        & 0.998         & 0.005         & 0.049           & 0.005            & 0.009            \\
		29         & 0.515        & 0.998         & 0.005         & 0.052           & 0.004            & 0.009            \\
		31         & 0.509        & 0.998         & 0.004         & 0.057           & 0.003            & 0.008            \\
		33         & 0.528        & 0.999         & 0.004         & 0.053           & 0.002            & 0.008            \\
		35         & 0.508        & 1.000         & 0.001         & 0.061           & 0.001            & 0.005            \\
		37         & 0.533        & 1.000         & 0.000         & 0.054           & 0.000            & 0.000            \\
		39         & 0.528        & 1.000         & 0.000         & 0.059           & 0.000            & 0.000            \\
		41         & 0.536        & 1.000         & 0.000         & 0.057           & 0.000            & 0.000            \\
		43         & 0.541        & 1.000         & 0.000         & 0.058           & 0.000            & 0.000            \\
		45         & 0.532        & 1.000         & 0.000         & 0.062           & 0.000            & 0.000            \\
		47         & 0.541        & 1.000         & 0.000         & 0.057           & 0.000            & 0.000            \\
		49         & 0.541        & 1.000         & 0.000         & 0.056           & 0.000            & 0.000            \\
		\cellcolor{green!25}51         &  \cellcolor{green!25}0.545        &  \cellcolor{green!25}1.000         &  \cellcolor{green!25}0.000         &  \cellcolor{green!25}0.053           &  \cellcolor{green!25}0.000            &  \cellcolor{green!25}0.000            \\
		53         & 0.539        & 1.000         & 0.000         & 0.056           & 0.000            & 0.000            \\
		55         & 0.539        & 1.000         & 0.000         & 0.057           & 0.000            & 0.000            \\
		57         & 0.535        & 1.000         & 0.000         & 0.056           & 0.000            & 0.000            \\
		59         & 0.545        & 1.000         & 0.000         & 0.049           & 0.000            & 0.000            \\
		61         & 0.539        & 1.000         & 0.000         & 0.052           & 0.000            & 0.000            \\
		63         & 0.540        & 1.000         & 0.000         & 0.052           & 0.000            & 0.000            \\
		65         & 0.545        & 1.000         & 0.000         & 0.045           & 0.000            & 0.000            \\
		67         & 0.534        & 1.000         & 0.000         & 0.052           & 0.000            & 0.000            \\
		69         & 0.540        & 1.000         & 0.000         & 0.047           & 0.000            & 0.000            \\
		71         & 0.543        & 1.000         & 0.000         & 0.047           & 0.000            & 0.000            \\
		73         & 0.532        & 1.000         & 0.000         & 0.053           & 0.000            & 0.000            \\
		75         & 0.541        & 1.000         & 0.000         & 0.044           & 0.000            & 0.000            \\
		77         & 0.541        & 1.000         & 0.000         & 0.042           & 0.000            & 0.000            \\
		79         & 0.538        & 1.000         & 0.000         & 0.044           & 0.000            & 0.000            \\
		81         & 0.529        & 1.000         & 0.000         & 0.050           & 0.000            & 0.000            \\
		83         & 0.537        & 1.000         & 0.000         & 0.048           & 0.000            & 0.000           \\ \hline
		\caption{Delinquency Model Analysis K-NN}
		\label{Delinquency Model Analysis K-NN}
	\end{longtable}
}  

ROC was used to select the optimal model using  the largest value.

The final value used for the model was k = 51.

\begin{table}[H]
	\centering\
	\resizebox{\textwidth}{!}
	{
		\begin{tabular}{l l r r r r}
			\hline
			\textbf{Model} &  \textbf{Dataset} & \textbf{\# Bad} & \textbf{\# Good} & \textbf{\# Observations} & \textbf{Good:Bad} \\
			\hline
			Previous Delinquency & Training Synthetic K51 & 1,449 & 1,932 & 3,381 & 50:50\\
			& Test & 245 & 633 & 878 & 72:28\\\hline
			\textbf{Previous Previous Delinquency}     & \textbf{Total} & \textbf{1,694} & \textbf{2,565} & \textbf{4,259} & \textbf{60:40} \\
			\hline
			No Previous Delinquency & Training Synthetic K75 & 9,480 & 18,012 & 27,492 & 66:34 \\ 
			& Test & 177 & 7,070 & 7,247 & 97:03 	\\\hline
			\textbf{No Previous Delinquency}     & \textbf{Total} & \textbf{9,597} & \textbf{25,032} & \textbf{34,739} & \textbf{72:28} \\
			\hline
			\textbf{Total } 	&     	     & \textbf{11,291} & \textbf{27,597} & \textbf{38,998} & \textbf{71:29}\\ \hline
		\end{tabular}
	}
	\caption{Breakdown Holdout Training/Test Dataset \\for Synthetic Benchmark Models}
	\label{table:benchmark_holdout_Synthetic_train_test}
\end{table}


\begin{table}[H]
	\centering
	\resizebox{\textwidth}{!}
	{
		\begin{tabular}{l l l r r r r}
			\hline
			\textbf{Model} & \textbf{Dataset} & \textbf{Software} & \textbf{Recall} & \textbf{Specificity} & \textbf{BA} & \textbf{AUC}  \\ \hline
			\textit{PD\_SyntheticBench\_R} & Previous Delinquency & R & 0.444 & 0.818 & 0.631 & 0.652   \\ \hline
			\textit{NPD\_SyntheticBench\_R} & No Previous Delinquency & R & 0.610 & 0.677 & 0.643 & 0.67   \\ \hline
		\end{tabular}
	}
	\caption{Benchmark Synthetic Model results for Comparison}
	\label{table:benchmodelSynthetic}
\end{table}

\begin{table}[H]
	\centering
	\small
	\begin{tabular}{l  r r r r}
		\hline
		\textbf{Model} & \textbf{Recall} & \textbf{Specificity} & \textbf{BA} & \textbf{AUC}  \\ \hline
		\textit{PD\_SyntheticExper\_R}  & 0.444 & 0.803 & 0.624 & 0.650   \\ \hline
		\textit{NPD\_SyntheticExper\_R} & 0.621 & 0.656 & 0.639 & 0.669   \\ \hline
	\end{tabular}
	\caption{Experiment Oversampling Model Results for Experiment Comparison}
	\label{table:benchmodelSyntheticExper}
\end{table}



\section{Conclusion}