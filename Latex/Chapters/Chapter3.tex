% Chapter Template

\chapter{Data} % Main chapter title

\label{Chapter3} % Change X to a consecutive number; for referencing this chapter elsewhere, use \ref{ChapterX}

\lhead{Chapter 3. \emph{Data}} % Change X to a consecutive number; this is for the header on each page - perhaps a shortened title

%----------------------------------------------------------------------------------------
%	SECTION 1
%----------------------------------------------------------------------------------------
\section{Introduction}
This chapter presents the data that will be used for the experiments to be carried out in this research and will be split into two main sections. 

The first section will outline where the customers for the experiment have been gathered from, and under what criteria they have been selected. As part of the experiment a baseline predictive model will be built, this will be done using features that were used in historic industry credit scorecard models in \subjectname. Performance measure measure will also selected evaluated also using this base line analysis later seen in Chapter \ref{Chapter4}.

The second section will outline what macro-economic features will be built as part of the experiment in this research. The aim of this research is to investigate the predictive power of macro-economic features by geographic regions such as electoral divisions and local authority in Ireland. To do this the addresses that are stored in \subjectname's databases are queried against an search engine and string metric algorithm application built for this research map addresses to \textit{global positioning system} (GPS) coordinates. The macro-economic features will be sourced and created from internal data sources in \subjectname\ and externally from open data sources. It will also explain how these features have been created, what transformations or data wrangling had to be done so the features fit/map into an analytical base table (ABT) discussed in Section \ref{sec:datasetConstruction} that is a requirement for predictive modelling.

\section{Experiment Set-up}
This section will detail data is in scope for prediction in this research. In Fig. \ref{fig:experiment_setup1} below SME customers are selected at the \textit{observation point}, June 2014. These customers are not in default at this point in time. Information prior to the observation point will be used for modelling to predict if a customer is likely to go into default, this is known as the \textit{performance window}.

Data from each individual customers performance will be taken from data in the performance window time period, which will be combined with macro location-based data prior to the observation point also. This data will be aggregated and structured into features for an ABT. The aim will be that these features will be able to distinguish what customers are likely to default on there repayment in the next 12 months. 

\begin{figure}[H]
	\includegraphics[width=0.8\textwidth,center]{experiment_setup}
	\caption[Experiment Performance Window and Outcome Window]
	{Experiment Performance Window and Outcome Window}
	\label{fig:experiment_setup1}
\end{figure}

Previous, Section \ref{classLabelDef} described that there were two methods used to define if a customer was in default or not: (i) the \textit{worst status} label definition method; and (ii) the \textit{current status} label method approach. As mentioned previously, for the purpose of this experiment we will be using the industry standard worst status method, this means if the customers is in 90+ days arrears at any stage in the outcome window they will be labelled as a bad customer or as one that has defaulted on their financial obligation. 


\section{Customers for Credit Scoring and Existing Features}

The customer data used for prediction in these experiments was sourced from a financial institution \subjectname, which is one of the two main pillar banks in Ireland. It contains details of 27,082 SME customers who were active between June 2014 and June 2015. These 27,082 customers are a subset of SME customers on the \subjectname\ book as the experiment will only be completed on one of the loan systems in the financial institution. None of the 27,082 customers are in default at the observation point(June 2014). The baseline model for this experiment will be built from features from a historic scorecard in \subjectname.

As mentioned in Section \ref{sec:segment} it is very common in credit scoring to model the population into multiple groups. This is done so homogeneous customers are grouped and modelled together based on for example pattern, characteristic, demographic etc. This common practice in industry also, in \subjectname\ is one method of modelling the credit risk of customers by splitting the customers into two segments. The criteria for selecting which segment each customer is modelled in is if that customer has been in arrears previously or not. For for this research a customer will be modelled in the \textit{Previous Delinquency} segment, if the customer \textbf{has not been} performing well and been in arrears previously. If the customer \textbf{has been } performing well and not been in arrears previously they will be modelled in the \textit{No Previous Delinquency} segment. The historic scorecards that were used in \subjectname\ used different features to build a model for each segment population. This was done using empirical analysis of the data where it was observed that different features contributed to the prediction of each subset of the population with a few overlapping features in each subset. Due to sensitivity of the information in \subjectname, the feature set for these models could not be documented in this research paper.

As mentioned in Section \ref{sec:imBalance} class imbalance in datasets is a major real world problem when building a predictive model. As mentioned previous this happens when the target class is is not distributed evenly in the dataset. The dataset in this experiment suffers from this imbalance also, however because the data is partitioned into two segments the imbalance in \textit{previous delinquency} dataset improves significantly but gets worse in the \textit{no previous delinquency} dataset.

The characteristics of the two datasets that will be used to build the baseline benchmark can be illustrated in Table \ref{characteristicsDatasets} below.

\begin{table}[H]
	\centering\
	\resizebox{\textwidth}{!}
	{
		\begin{tabular}{l r r r r r r}
			\hline
			\textbf{Model} &  \textbf{\# Numeric} & \textbf{\# Nominal} & \textbf{\# Observations} & \textbf{\# Good} & \textbf{\# Bad} & \textbf{Good:Bad}\\
			\hline
			Previous Delinquency & 11 & 0 & 2,926 & 2,198  & 738 & 75:25 \\ 
			No Previous Delinquency & 9 & 0 & 24,156 & 23,505 & 651  & 97:03 \\ \hline
			\textbf{Total} &  &  & \textbf{27,082} & \textbf{25,703} & \textbf{1,389} & \textbf{95:05} \\ \hline
		\end{tabular}
	}
	\caption{Characteristics of datasets to be used in the exploratory evaluation for training a baseline model and assessing the evaluation metrics to be used in the research \\
		 \# Numeric refers to the number of continuous features \\
		 \# Nominal refers to the number of categorical features
		}
	\label{characteristicsDatasets}
\end{table}

It can be seen above in Table \ref{characteristicsDatasets} that two datasets are not very similar. There are only 2,936 customers in the \textit{previous delinquency} dataset and 24,156 in the \textit{no previous delinquency} dataset. Perhaps the the biggest difference is class imbalance difference between the datasets. 25\% of customers in the \textit{previous delinquency} dataset are bad by the end of the outcome window but there is only 3\% of the \textit{no previous delinquency} dataset that are bad at the end of the outcome window. As discussed in Section \ref{sec:imBalance} this presents significant challenges when try to build a predictive model but also a very common challenge in real world applications. 


\section{Macro-Economic Areas for Experiment}
The experiment in this research is to investigate if macro-economic features by locations are useful for predicting if SME customers are likely to go into default. The two macro-economic regions that features will be based on are \textit{Electoral Division} (ED) and \textit{Local Authority} (LD). Below Fig. \ref{fig:Ireland_ED_LA_Example} maps out the electoral divisions and local authorities in the Republic of Ireland.

\begin{figure}[H]
	\begin{subfigure}[b]{0.5\textwidth}
		\captionsetup{font=scriptsize}
		\includegraphics[width=\textwidth,height = 10cm]{IrelandElectoralDivisions}
		\caption{Map of the Current 3,440 Electoral Divisions}\label{fig:IrelandElectoralDivisions}
	\end{subfigure} ~\quad
	\begin{subfigure}[b]{0.5\textwidth}
		\captionsetup{font=scriptsize}
		\includegraphics[width=\textwidth,height = 10cm]{IrelandLocalAuthorities}
		\caption{Map of the Current 34 Local Authorities}\label{fig:IrelandLocalAuthorities}
	\end{subfigure}
	\caption{Republic of Ireland's Electoral Divisions Local Authorities}
	\label{fig:Ireland_ED_LA_Example}
\end{figure}

There are 34 primary local authorities in the Republic of Ireland, including 29 county councils and 5 city councils. Organisations within each area are responsible for managing issues such as housing, planning, roads, water supply and sewerage, development incentives and controls, environmental protection, recreation facilities and amenities, agriculture, education, health and welfare \footnote{\url{http://www.iro.ie/local_authorities.html}}. There are 3,440 electoral divisions in the Republic of Ireland. Electoral divisions are formed by grouping town-lands together and are the smallest legally defined administrative areas in the state which small population statistics are published from in the Census \footnote{\url{http://census.cso.ie/censusasp/saps/boundaries/eds_bound.htm}}.


\section{Converting Addresses to GPS Coordinates}
To link macro-economic features by electoral division and local authorities to a SME customer there has to be a mechanism. In an ideal world this would be done through a data model where a customers address would be linked to the address reference database. This can be illustrated through the Entity-Relationship Model (ERD) in Fig. \ref{fig:MasterReferenceAddressDataArchitecture}.

\begin{figure}[H]
	\includegraphics[width=1\textwidth,center]{MasterReferenceAddressDataArchitecture}
	\caption{Optimal Entity Relationship Data Model for Mapping Customers, Addresses, Electoral Division and Local Authorities}
	\label{fig:MasterReferenceAddressDataArchitecture}
\end{figure}

Unfortunately this is not the case currently in \subjectname. Currently addresses are stored in free text fields across multiple systems. This can be illustrated in Fig. \ref{fig:AIB_Situtation} where you can see unstandardised and free text addresses are stored in the Customer table instead of an address reference table like in Fig. \ref{fig:MasterReferenceAddressDataArchitecture}.

\begin{figure}[H]
	\includegraphics[width=.7\textwidth,center]{AIB_Situtation}
	\caption{Current Entity Relationship Data Model for Mapping Customers, Addresses, Electoral Division and Local Authorities}
	\label{fig:AIB_Situtation}
\end{figure}

With the release of Ireland's new postcode system in July 2015 Eircode\footnote{\url{http://www.eircode.ie/}} \subjectname\ was looked to position itself strongly for its deployment and how it would integrate into \subjectname's current systems. For this there and for other private reason there was an investment in the GeoDirectory database \footnote{\url{https://www.geodirectory.ie/}}. It is a product established by An Post\footnote{\url{http://www.anpost.ie/AnPost/}} and the Ordnance Survey Ireland\footnote{\url{http://www.osi.ie/}}. It provides a complete database of all the addresses in the Republic of Ireland and geolocation details including 1.8 million buildings. The Eircode database is heavily connected with GeoDirectry database as it is essentially the same database with a new address identifier called \textit{Eircode}. The database diagram for GeoDirectory is show below in Fig. \ref{fig:GeoDirectoryDatabase} which includes an Electoral Division and Local Authorities table.

\begin{figure}[H]
	\includegraphics[width=1.2\textwidth,center]{GeoDirectoryDatabase}
	\caption{GeoDirectory database diagram}
	\label{fig:GeoDirectoryDatabase}
\end{figure}

So to complete this research and experiment a mechanism or application for matching customer addresses to the the GeoDirectory database needed to be built. There are many vendors in Ireland and internationally that provide services to correct and validate addresses such as Address Doctor\footnote{\url{https://www.informatica.com/addressdoctor.html}}, Gamma\footnote{\url{http://www.gamma.ie/about-gamma}} and Data Ireland\footnote{\url{http://www.dataireland.ie/}} to name a few. Committing to one of these products would require a project to evaluate each service where \subjectname\ would analyse the pros and cons, understanding the full requirements of the financial institution.

In the interim as part of this research it was decided to look at in house solution that could be developed using existing resources and open source technologies. After some investigation and experimentation it was identified that it was possible to build a solution levering an address database GeoDirectory database, a search platform/engine Solr\footnote{\url{http://lucene.apache.org/solr/}} and a high level programming language Python \footnote{\url{https://www.python.org/}}.


Solr search works by creating an index of the data chosen for application. An example of this can be illustrated in Fig. \ref{fig:solrIndexing}\footnote{\url{http://blog.e-zest.net/about-apache-solr/}}. 

\begin{figure}[H]
	\includegraphics[width=1\textwidth,center]{solrIndexing}
	\caption[Illustration of Inverted Indexing]
	{How Solr Indexes and Stores Data}
	\label{fig:solrIndexing}
\end{figure}

For this experiment the GeoDirectory database was indexed using Solr allowing it to be queried through the web interface or multiple Application Program Interface (API) such as Python, JavaScript, Ruby, Java, HTTP to name a few. Solr returns a number results from queries posted against it based on these indexes which can be seen in Fig. \ref{fig:Solr_Example_Search from Web Interface}.

\begin{figure}[H]
	\includegraphics[width=1.1\textwidth,,height = 10cm,center]{Solr_Example_Search}
	\caption{Solr Query Example and Syntax}
	\label{fig:Solr_Example_Search from Web Interface}
\end{figure}

Although this solution worked quite well in the majority of cases observed it did have some issues because way Irish addresses are structured. For example because there are number of addresses that contain ``Some County Road'' there were cases when the first result returned by Solr returned a false positive. To cater for this number of the top results returned from Solr were compared using sting similarity metrics. String comparison algorithms/metrics are used to determine the distance or number of changes between two strings \citep{wagner_string--string_1974}. The two string comparison methods that were used as part of this experiment are the \textit{Levenshtein Distance} \citep{levenshtein_binary_1966} and \textit{Jaro–Winkler Distance} \citep{winkler_string_1990}. The Levenshtein is computed by calculating the smallest number of single character changes between to strings. The score can be normalised for it so it produces a value between 0 and 1 by $1 -\frac{\text{number of edits}}{\text{length of the larger string}}$. It is very useful for compensating for typos in string matching. The Jaro-Winkler algorithm is used to measure number of characters in common but also works on the basis that differences at the start of the string are more important than those at end. In research completed by \cite{christen_comparison_2006} the Jaro-Winkler method techniques performed quite well across all experiments and was included in this experiment as a result. 

Fig. \ref{fig:address_matching_application} illustrates the full application that was built as part of this research to map addresses in \subjectname\ to a master address database, combining GeoDirectory, Solr and Python.

\begin{figure}[H]
	\includegraphics[width=.8\textwidth,center]{address_matching_application}
	\caption{Address Matching Application for this Experiment}
	\label{fig:address_matching_application}
\end{figure}

From test cases it was observed that the application appears to work quite especially at matching addresses to Electoral Divisions and Local Authorities which is vital for this research. There are some inaccuracies due to data quality issues with the originating address but that was only when mapping to one specific address in the GeoDirectory database not electoral division or local authority.

Unfortunately there was not scope in this research to carry out further analysis on the accuracy of the results but the business were so impressed with the results they observed that they are going to carry out an enterprise product investigation and evaluation. Leading on from this research they will be taking a sample of 20,000 addresses in \subjectname\ and allowing vendors detailed earlier to return their results. These results will then be collated by the business. Testing will done using a crowd sourcing to build confidence intervals to evaluate which product and service offered the best result. An informed data driven decision can then be made if a proprietary address matching application is needed or the application from this research is accurate and meets the requirements of \subjectname.  


\section{Data for Experiment}
As discussed in the literature there has been much evidence to support the idea that macro-economic trends such as unemployment and arrears rates at a regional areas are useful features for predicting credit risk and future unperforming SME customers.

The main experiments to be carried out as part of this research will aim to investigate the macro-economic features can improve the prediction model in \subjectname\ compared to the results of prediction model based on application and customer behavioural features from a historic scorecard. 

There will be 5 main categories of experimental features created and tested as part of this experiment which will be explained in the next subsections of this chapter. 

\begin{itemize}
	\item Personal customer card spending behaviour
	\item Feature grouping based on Home Loans, Personal Loans, SME Loans
	\item Central Statistics Office (CSO) Features
	\item SME default behaviour
	\item Personal Loans \& Homeloans default behaviour
\end{itemize}


\subsubsection{Personal customer card spending behaviour}
One experiment in this research will test to see if customers transactional spending behaviour metrics could be useful for predicting if SME customers will default. The intuition for this analysis is that if customers in one area are suffering hardship there spending habits might be reflective of that and as consumers this will have an adverse affect on SMEs business. To do this transactions will be gathered from the personal customers VISA debit card transactional database in \subjectname. 192 million transactions from 1.3 million customers  will be collected for 12 month period prior to the observation point (June 2014) in the experiment. 

Merchants are assigned what is known as a merchant category code (MCC), it is used generally for classifying the primary business of the merchant \footnote{\url{https://www.visa.com/supplierlocator/}}. These codes have been leveraged in \subjectname\ to create a Money Manager Application which rolls these MCC in MCC categories. This application allows personal customers to keep track of their spending behaviour through a combination of reports, visualisation and search functionality. These categories are broken down into parent and child categories for this application. These categories can be seen in Table \ref{table:mcc} below

\begin{table}[H]
	\centering
	\resizebox{\textwidth}{!}
	{
		\label{my-label}
		\begin{tabular}{|l|l|l|l|}
			\hline
			\textbf{MCC Category} & \textbf{Parent}          & \textbf{Child}                 & \textbf{Spend / Live} \\ \hline
			2.2                   & Bills \& Utilities       & Cable/Satellite TV \& Internet & Spend                 \\ \hline
			2.7                   & Bills \& Utilities       & Gas/Electricity/Energy         & Live                  \\ \hline
			3.1                   & Leisure \& Entertainment & Cinema \& Theatre              & Spend                 \\ \hline
			4.1                   & Shopping                 & Groceries                     & Live                  \\ \hline
			4.4                   & Shopping                 & Clothing \& Accessories        & Spend                 \\ \hline
			5.2                   & Health \& Personal Care  & Doctor                         & Live                  \\ \hline
			5.6                   & Health \& Personal Care  & Hair \& Beauty                 & Spend                 \\ \hline
			6.2                   & Household/Home           & Household Maintenance          & Live                  \\ \hline
			6.5                   & Household/Home           & Computers \& Technology        & Spend                 \\ \hline
		\end{tabular}
	}
	\caption{MCC Categories and Spend/Live Categorisation Sample}
	\label{table:mcc}
\end{table}

For this research and experiment these transactions MCC categories have been assigned \textit{Spend} or \textit{Live} category. Transactions categorised as \textit{Spend} will translate to what has been derived as discretionary spend of \subjectname\ customers. This will include transactions such as social activities such as going to the cinema, going to bars and clubs and eating out in restaurants which can be seen in Table \ref{table:mcc}. Transactions categorised as \textit{Live} will translate to what as been derived as transactions required/associated/needed for customers to live. This will include transactions such as paying bills, shopping/groceries, healthcare etc.

Transactions for each customer will be aggregated to electoral division and local authority for each month for the 12 months prior to the observation point. Metrics will be created calculating the difference between time periods to evaluate what trends or patterns are useful for identifying default. 

The rates of change between transactional spend will be calculated by using the percentage change equation seen below

\begin{equation} \label{eqn:PercentageChange}
\text{Percentage Change} = \frac{X^{2} - X^{1}}{X^{1}}*100  
\end{equation}
Where: 
\vspace{-7mm} 
\begin{itemize}
	\item $X^{1} = $ the original variable
	\item $X^{2} = $ the new variable
\end{itemize}

These features will not be normalised so feature rescaling will be applied to ensure the features range value is between 0 and 1, this is down through applying the rescaling equation found below  

\begin{equation}\label{eqn:rescaling}
\text{Feature Scaling} =  X' = \frac{X - X_{min}}{X_{max} - X_{min}}
\end{equation}

There will be 68 features created as part of this experiment. The names, data types and description for each features can be found in Appendix \ref{AppendixA}.



\subsubsection{Personal Loans \& Homeloans default behaviour}
An experiment will be carried out as part of this research to investigate if \subjectname's personal customers default ratios of loans and homeloans to investigate if they could be useful in predicting if SME customers will default. The intuition for this insight is that if there is a high proportion of personal loan and homeloan defaulting in an area then there could be a relationship with SMEs also defaulting. It was also discussed in literature by \cite{di_pietro_regional} how metrics like this were useful for predicting default in Italy.

At the observation point (June 2014), personal loan and homeloan products that were active on the \subjectname\ book were analysed. Table \ref{table:AIB_Products} details the products that were extracted as part of this experiment.

\begin{table}[H]
	\centering
		\small
			\begin{tabular}{|l|l|l|l|}
			\hline
			\textbf{Product Category} & \textbf{Product Name} \\ \hline
			Branch Advances     & Fixed Loans \\ \hline
			Branch Advances     & Matrix Loans \\ \hline
			Branch Advances     & Other Loans \\ \hline
			Branch Advances     & Premium Business Rate \\ \hline
			Branch Advances     & Prime Loans \\ \hline
			Branch Advances     & Staff Credit Flex \\ \hline
			Branch Advances		& Suspense Interest \\ \hline
			Home Loan           & Buy to Let \\ \hline
			Home Loan           & Commercial Mortgages \\ \hline
			Home Loan           & Home Loan \\ \hline
			Home Loan           & Staff Homeflex \\ \hline
			Home Loan           & Standard Mortgages \\ \hline	
			Home Loan           & Surplus Builder \\ \hline		
			Home Loan           & Tracker Mortgages \\ \hline		
		\end{tabular}
	\caption{\subjectname\ Personal Loan and Homeloan products}
	\label{table:AIB_Products}
\end{table}

Each product in the Table \ref{table:AIB_Products} above will be rolled up to its product category. Each product will be associated with customer and customer to an address so they can be aggregated to electoral division and local authority. The products are then classified as default or not default so a ratio can be built for each electoral division and local authority. Fig. \ref{fig:BothPersonalRatio} below illustrates the default rates for each local authority for personal loan and homeloan products combined. 

\begin{figure}[H]
	\includegraphics[width=1\textwidth,center]{BothPersonalRatio}
	\caption{Default Ratios for Homeloan and Personal Loans by Local Authority}
	\label{fig:BothPersonalRatio}
\end{figure}

There will be 6 features created as part of this experiment. The names, data types and description for each features can be found in Appendix \ref{AppendixD}.


\subsubsection{Central Statistics Office (CSO) Features}
An experiment will be carried out as part of this research to investigate if data collected as part of the Irish census is indicative of SME default. Census are carried out every 5 years in the Republic of Ireland to paint a picture of living and social conditions of the population. It provides details to the smallest geographic areas which can be used for decision making and planning. For example to find what infrastructure and services need to be invested in each areas for example schools, job training centre and health care services\footnote{\url{http://census.ie/}}. 

The data from the census can be organised into 11 themes which can be illustrated below in Table \ref{table:censusThemes}

\begin{table}[H]
	\centering
	\small
	\begin{tabular}{|l|l|}
		\hline
		\textbf{Theme Number} & \textbf{Theme Description} \\ \hline
		Theme 1    & Sex, age and marital status \\ \hline
		Theme 2    & Migration, ethnicity and religion \\ \hline
		Theme 3    & Irish Language \\ \hline
		Theme 4    & Families \\ \hline
		Theme 5    & Private Household \\ \hline
		Theme 6    & Housing \\ \hline
		Theme 7	   & Communal establishments \\ \hline
		Theme 8    & Principal status \\ \hline
		Theme 9    & Social class and socio-economic group \\ \hline
		Theme 10   & Education \\ \hline
		Theme 11   & Commuting \\ \hline	
		Theme 12   & Disability, careers and general health \\ \hline		
		Theme 13   & Occupation \\ \hline
		Theme 14   & Industries \\ \hline
		Theme 15   & PC and internet Access \\ \hline			
	\end{tabular}
	\caption{Themes of data available in the Irish census 2011}
	\label{table:censusThemes}
\end{table}


Anecdotally low unemployment rates and higher education rates would be associated with areas good economic growth and prosperity. Since the financial crisis of 2008-2009 there has been huge reduction in the number of construction projects which is said to be heavily linked to higher unemployment. As a result for this experiment features have been created using data provided from the census at local authority and electoral division areas. The features that have been created are lower than secondary level of eduction ratio, manual based occupation ratio, unemployment ratio by electoral division and local authority. There was some data manipulation and business logic applied to transform the raw data into these metrics.

There will be 6 features created as part of this experiment. The names, data types and description for each features can be found in Appendix \ref{AppendixE}.



\subsubsection{SME Default behaviour}
An experiment will be carried out as part of this research to investigate if SME default ratios and behavioural changes could be useful in predicting future SME defaults. At an anecdotal you could think of this as domino effect where businesses near each other close in sequence due to economic hardship in that area and low consumer confidence. For example one feature could be have default rates increased or decreased for SME customers at an electoral division or local authority area. It was also discussed in literature by \cite{di_pietro_regional} how this was useful for predicting default in Italy. 

The data for this experiment was extracted from the SME database in \subjectname\ from June 2012 to the June 2014(Observation Point). Each SME customers local authority and electoral division was identified, and customers were classified as in default or not in default for each month in the two year period. Features were then built with a focus of identifying local authorities and electoral division which had seen higher default rates in the two years prior to the observation point. 

The majority of the features will be built using the percentage change formula seen above Equation \ref{eqn:PercentageChange} and the rescaling formula Equation \ref{eqn:rescaling}. But it will also include the ratio between default and non-default, and count of default for each local authority and electoral division.

There will be 20 features created as part of this experiment. The names, data types and description for each features can be found in Appendix \ref{AppendixF}.

\subsubsection{Feature grouping based on Home Loans, Personal Loans, SME Loans}
As reviewed in the literature in Section \ref{sec:binning}, binning is a useful method of transforming interval based features into categorical features. This has a number of benefits such as simplifying the structure of the data into nominal and ordinal based features and interval based on estimating the groups weight of evidence (WoE). Binning increases scoring models robustness by generalising model for unseen data thus reducing the chances of over-fitting. As discussed in the literature it also has the capability to incorporate missing values and other extreme outliers which cause instability in the model. 

The features to be binned as part of this experiment are based on features already seen in this section. Homeloan, personal loan and SME default ratios at the observation point (June 2014) will be binned in attempt to build robust and simplified feature set. This experiment will test the nominal/ordinal outputs and WoE outputs in the predictive modelling stage to see if either improve the prediction of the baseline model.

There will be 16 features created as part of this experiment. The names, data types and description for each features can be found in Appendix \ref{AppendixG}.


\section{Conclusions}\label{desConc}


\begin{comment}
This chapter has discussed the processes required to carry out data mining techniques on non-relational web log data. The required data transformations have been discussed as well as the implemented methods for feature generation.
\end{comment}