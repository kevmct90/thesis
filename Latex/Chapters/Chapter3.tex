% Chapter Template

\chapter{Data} % Main chapter title

\label{Chapter3} % Change X to a consecutive number; for referencing this chapter elsewhere, use \ref{ChapterX}

\lhead{Chapter 3. \emph{Data}} % Change X to a consecutive number; this is for the header on each page - perhaps a shortened title

%----------------------------------------------------------------------------------------
%	SECTION 1
%----------------------------------------------------------------------------------------
\section{Introduction}
This chapter presents the data that will be used for the experiments to be carried out in this research and will be split into two main sections. 

The first section will outline where the customer data for the experiment have been gathered from, and under what criteria they have been selected. As part of the experiment a baseline predictive model will be built, this will be done using features that were used in historic industry credit scorecard models in \subjectname. The performance measure to evaluate the experiment will also selected evaluated using the baseline model.

The second section will outline what macro-economic features will be built as part of the experiment in this research. The aim of this research is to investigate the predictive power of macro-economic features by geographic regions such as electoral divisions and local authority in Ireland. To do this the addresses that are stored in \subjectname's databases are queried against a search engine and string metric algorithm application to map customers addresses to \textit{global positioning system} (GPS) coordinates. The macro-economic features will be sourced and created from internal data sources in \subjectname\ and externally from open data sources. This section will also explain how these features have been created, what transformations or data wrangling was required so the features fit/map into an analytical base table (ABT) discussed in Section \ref{sec:datasetConstruction}. Care was also taken throughout this experiment to ensure that anachronistic variables were not included in any of the predictive models built as part of this research e.g. Features that contain information about the outcome after the observation point. 

\section{Experiment Set-up}
This section will detail the data that is in scope for prediction in this research. In Fig. \ref{fig:experiment_setup1} below SME customers are selected at the \textit{observation point}, June 2014. These customers are not in default at this point in time. Information prior to the observation point will be used for modelling to predict if a customer is likely to go into default, this is known as the \textit{performance window}.

Data from each individual customers performance will be taken from data in the performance window, which will be combined with macro location-based data prior to the observation point. This data will be aggregated and structured into features for an ABT. The aim will be that these features will be able to distinguish what customers are likely to default on there repayment in the next 12 months. 

\begin{figure}[H]
	\includegraphics[width=0.8\textwidth,center]{experiment_setup}
	\caption[Experiment Performance Window and Outcome Window]
	{Experiment Performance Window and Outcome Window}
	\label{fig:experiment_setup1}
\end{figure}

Previously in Section \ref{classLabelDef} described that there are two methods used to define if a customer was in default or not: (i) the \textit{worst status} label definition method; and (ii) the \textit{current status} label method approach. As mentioned previously, for the purpose of this experiment we will be using the industry standard worst status method, this means if the customers is in 90+ days arrears at any stage in the outcome window they will be labelled as a bad customer or as one that has defaulted on their financial obligation. 


\section{Customers for Credit Scoring and Existing Features}\label{sec:existFeatures}

The customer data used for prediction in these experiments was sourced from a financial institution \subjectname, which is one of the two main pillar banks in Ireland. It contains details of 27,082 SME customers who were active between June 2014 and June 2015. These 27,082 customers are a subset of SME customers on the \subjectname\ book as the experiment will only be completed on one of the loan systems in the financial institution. The customers default status at June 2015 will be recorded and used as the target for building the prediction model. The baseline model for this experiment will be built from features from a historic scorecard in \subjectname.

As mentioned in Section \ref{sec:segment} it is very common in credit scoring to model the population into multiple groups. This is done so homogeneous customers are grouped and modelled together based on for example pattern, characteristic, demographic etc. This is common practice in industry also, in \subjectname\ one method of modelling the credit risk of customers is by splitting the customers into two segments. The criteria for selecting which segment each customer is modelled in is if that customer has been in arrears previously or not. For for this research a customer will be modelled in the \textit{Previous Delinquency} segment, if the customer \textbf{has not been} performing well and been in arrears previously. If the customer \textbf{has been} performing well and not been in arrears previously they will be modelled in the \textit{No Previous Delinquency} segment. The historic scorecards that were used in \subjectname\ used different features to build a model for each segment population. This was done using empirical analysis of the data where it was observed that different features contributed to the prediction of each subset of the population with a few overlapping features in each subset. Due to sensitivity of the information in \subjectname, the feature set for these models will not be documented in this research paper.

As mentioned in Section \ref{sec:imBalance} class imbalance in datasets is a major real world problem when building a predictive model. This happens when the target class is not distributed evenly in the dataset. The dataset in this experiment suffers from this imbalance also, however because the data is partitioned into two segments the imbalance in \textit{previous delinquency} dataset improves significantly but gets worse in the \textit{no previous delinquency} dataset.

The characteristics of the two datasets that will be used to build the baseline benchmark are illustrated in Table \ref{characteristicsDatasets} below.

\begin{table}[H]
	\centering\
	\resizebox{\textwidth}{!}
	{
		\begin{tabular}{l r r r r r r}
			\hline
			\textbf{Model} &  \textbf{\# Numeric} & \textbf{\# Nominal} & \textbf{\# Observations} & \textbf{\# Good} & \textbf{\# Bad} & \textbf{Good:Bad}\\
			\hline
			Previous Delinquency & 11 & 0 & 2,926 & 2,198  & 738 & 75:25 \\ 
			No Previous Delinquency & 9 & 0 & 24,156 & 23,505 & 651  & 97:03 \\ \hline
			\textbf{Total} &  &  & \textbf{27,082} & \textbf{25,703} & \textbf{1,389} & \textbf{95:05} \\ \hline
		\end{tabular}
	}
	\caption{Characteristics of datasets to be used in the exploratory evaluation for training a benchmark model and assessing the evaluation metrics to be used in the research \\
		 \# Numeric refers to the number of continuous features \\
		 \# Nominal refers to the number of categorical features
		}
	\label{characteristicsDatasets}
\end{table}

It can be seen above in Table \ref{characteristicsDatasets} that the two datasets are not very similar. There are only 2,936 customers in the \textit{previous delinquency} dataset and 24,156 in the \textit{no previous delinquency} dataset. Perhaps the the biggest difference is the class imbalance difference between the datasets. 25\% of customers in the \textit{previous delinquency} dataset are bad by the end of the outcome window but there is only 3\% of the \textit{no previous delinquency} dataset that are bad at the end of the outcome window. As discussed in Section \ref{sec:imBalance} this presents significant challenges when trying to build a predictive model but is also a very common challenge in real world applications. 


\section{Macro-Economic Areas for Experiment}
The experiment in this research is to investigate if macro-economic features by locations are useful for predicting if SME customers are likely to go into default. The two macro-economic regions that features will be based on are \textit{Electoral Division} (ED) and \textit{Local Authority} (LD). Below Fig. \ref{fig:Ireland_ED_LA_Example} maps out the electoral divisions and local authorities in the Republic of Ireland.

\begin{figure}[H]
	\begin{subfigure}[b]{0.5\textwidth}
		\captionsetup{font=scriptsize}
		\includegraphics[width=\textwidth,height = 10cm]{IrelandElectoralDivisions}
		\caption{Map of the Current 3,440 Electoral Divisions}\label{fig:IrelandElectoralDivisions}
	\end{subfigure} ~\quad
	\begin{subfigure}[b]{0.5\textwidth}
		\captionsetup{font=scriptsize}
		\includegraphics[width=\textwidth,height = 10cm]{IrelandLocalAuthorities}
		\caption{Map of the Current 34 Local Authorities}\label{fig:IrelandLocalAuthorities}
	\end{subfigure}
	\caption{Republic of Ireland's Electoral Divisions Local Authorities}
	\label{fig:Ireland_ED_LA_Example}
\end{figure}

There are 34 primary local authorities in the Republic of Ireland, including 29 county councils and 5 city councils. Organisations within each area are responsible for managing issues such as housing, planning, roads, water supply and sewerage, development incentives and controls, environmental protection, recreation facilities and amenities, agriculture, education, health and welfare\footnote{\url{http://www.iro.ie/local_authorities.html}}. There are 3,440 electoral divisions in the Republic of Ireland. Electoral divisions are formed by grouping town-lands together and are the smallest legally defined administrative areas in the state which small population statistics are published from in the Census\footnote{\url{http://census.cso.ie/censusasp/saps/boundaries/eds_bound.htm}}.


\section{Converting Addresses to GPS Coordinates}
To link macro-economic features by electoral division and local authorities to a SME customer there has to be a mechanism. In an ideal world this would be done through a data model where a customers address would be linked to the address reference database. This can be illustrated through the Entity-Relationship Model (ERD) in Fig. \ref{fig:MasterReferenceAddressDataArchitecture}.

\begin{figure}[H]
	\includegraphics[width=1\textwidth,center]{MasterReferenceAddressDataArchitecture}
	\caption{Optimal Entity Relationship Data Model for Mapping Customers, Addresses, Electoral Division and Local Authorities}
	\label{fig:MasterReferenceAddressDataArchitecture}
\end{figure}

Unfortunately this is not the case currently in \subjectname. Currently addresses are stored in free text fields across multiple systems. This can be illustrated in Fig. \ref{fig:AIB_Situtation} where you can see unstandardised and free text addresses are stored in the customer table instead of in an address reference table like in Fig. \ref{fig:MasterReferenceAddressDataArchitecture}.

\begin{figure}[H]
	\includegraphics[width=.7\textwidth,center]{AIB_Situtation}
	\caption{Current Entity Relationship Data Model for Mapping Customers, Addresses, Electoral Division and Local Authorities}
	\label{fig:AIB_Situtation}
\end{figure}

With the release of Ireland's new postcode system in July 2015 Eircode\footnote{\url{http://www.eircode.ie/}} \subjectname\ has looked to position itself strongly for its deployment and how it would integrate into \subjectname's current legacy systems. For this reason an investment was made in the GeoDirectory database \footnote{\url{https://www.geodirectory.ie/}}. It is a product established by An Post\footnote{\url{http://www.anpost.ie/AnPost/}} and the Ordnance Survey Ireland\footnote{\url{http://www.osi.ie/}}. It provides a complete database of all the addresses in the Republic of Ireland and geolocation details including 1.8 million buildings. The Eircode database is heavily connected with GeoDirectry database as it is essentially the same database with a new address identifier called \textit{Eircode}. The database diagram for GeoDirectory is shown below in Fig. \ref{fig:GeoDirectoryDatabase} which includes an Electoral Division and Local Authorities table.

\begin{figure}[H]
	\includegraphics[width=1.2\textwidth,center]{GeoDirectoryDatabase}
	\caption{GeoDirectory database diagram}
	\label{fig:GeoDirectoryDatabase}
\end{figure}

So to complete this research a mechanism or application for matching customer addresses to the the GeoDirectory database needed to be built. There are many vendors in Ireland and internationally that provide services to correct and validate addresses such as Address Doctor\footnote{\url{https://www.informatica.com/addressdoctor.html}}, Gamma\footnote{\url{http://www.gamma.ie/about-gamma}} and Data Ireland\footnote{\url{http://www.dataireland.ie/}} to name a few. Committing to one of these products would require a project to evaluate each service where \subjectname\ would analyse the pros and cons, understanding the full requirements of the financial institution.

In the interim as part of this research it was decided to look at in house solutions that could be developed using existing resources and open source technologies. After some investigation and experimentation it was identified that it was possible to build a solution leveraging an address database GeoDirectory, a search platform/engine Solr\footnote{\url{http://lucene.apache.org/solr/}} and a high level programming language Python \footnote{\url{https://www.python.org/}}.


Solr search works by creating an index of the data chosen for application. An example of this is illustrated in Fig. \ref{fig:solrIndexing}\footnote{\url{http://blog.e-zest.net/about-apache-solr/}}. 

\begin{figure}[H]
	\includegraphics[width=1\textwidth,center]{solrIndexing}
	\caption[Illustration of Inverted Indexing]
	{How Solr Indexes and Stores Data}
	\label{fig:solrIndexing}
\end{figure}

For this experiment the GeoDirectory database was indexed using Solr allowing it to be queried through the web interface or multiple Application Program Interface (API) such as Python, JavaScript, Ruby, Java, HTTP to name a few. Solr returns a number results from queries posted against it based on these indexes which can be seen in Fig. \ref{fig:Solr_Example_Search from Web Interface}.

\begin{figure}[H]
	\includegraphics[width=1.1\textwidth,,height = 10cm,center]{Solr_Example_Search}
	\caption{Solr Query Example and Syntax}
	\label{fig:Solr_Example_Search from Web Interface}
\end{figure}

Although this solution worked quite well in the majority of cases observed it did have some issues because of the way Irish addresses are structured. For example because there are a number of addresses that contain ``Some County Road'' there were cases when the first result returned by Solr returned a false positive. To cater for this, a number of the top results returned from Solr were compared using sting similarity metrics. String comparison algorithms/metrics are used to determine the distance or number of changes between two strings \citep{wagner_string--string_1974}. The two string comparison methods that were used as part of this experiment were the \textit{Levenshtein Distance} \citep{levenshtein_binary_1966} and \textit{Jaro–Winkler Distance} \citep{winkler_string_1990}. The Levenshtein is computed by calculating the smallest number of single character changes between two strings. The score can be normalised so it produces a value between 0 and 1 by $1 -\frac{\text{number of edits}}{\text{length of the larger string}}$. It is very useful for compensating for typos in string matching. The Jaro-Winkler algorithm is used to measure number of characters in common but also works on the basis that differences at the start of the string are more important than those at end. In research completed by \cite{christen_comparison_2006} the Jaro-Winkler method techniques performed quite well across all experiments and was included in this experiment as a result. 

Fig. \ref{fig:address_matching_application} illustrates the application that was built as part of this research to map addresses in \subjectname\ to a master address database, combining GeoDirectory, Solr and Python.

\begin{figure}[H]
	\includegraphics[width=.8\textwidth,center]{address_matching_application}
	\caption{Address Matching Application for this Experiment}
	\label{fig:address_matching_application}
\end{figure}

From test cases it was observed that the application appears to work quite well especially at matching addresses to Electoral Divisions and Local Authorities which is vital for this research. There are some inaccuracies due to data quality issues with the originating address but that was only when mapping to one specific address in the GeoDirectory database not electoral division or local authority.

Unfortunately there was not scope in this research to carry out further analysis on the accuracy of the results but the business were so impressed with the results they observed that they are going to carry out an enterprise product investigation and evaluation. Leading on from this research they will be taking a sample of 20,000 addresses in \subjectname\ and allowing vendors detailed earlier to return their results. These results will then be collated by the business. Testing will done using crowd sourcing to build confidence intervals to evaluate which product and service offered the best result. An informed data driven decision can then be made if a proprietary address matching application is needed or the application from this research is accurate and meets the requirements of \subjectname.  


\section{Data for Experiment}\label{sec:dataForExper}
As discussed in the literature there has been much evidence to support the idea that macro-economic trends such as unemployment and arrears rates at a regional areas are useful features for predicting credit risk and future unperforming SME customers.

The main experiments to be carried out as part of this research will aim to investigate if macro-economic features can improve the prediction model in \subjectname\ compared to the results of prediction model based on application and customer behavioural features from the historic scorecard. 

There will be 5 main categories of experimental features created and tested as part of this experiment which will be explained in the next subsections of this chapter. 

\begin{itemize}
	\item Personal customer card spending behaviour
	\item Feature grouping based on Home Loans, Personal Loans, SME Loans
	\item Central Statistics Office (CSO) Features
	\item SME default behaviour
	\item Personal Loans \& Homeloans default behaviour
\end{itemize}


\subsubsection{Personal customer card spending behaviour}
One experiment in this research will test to see if customers transactional spending behaviour metrics could be useful for predicting if SME customers will default. The intuition for this analysis is that if customers in one area are suffering hardship there spending habits might be reflective of that and as consumers this will have an adverse affect on SMEs businesses. To do this transactions will be gathered from the personal customers VISA debit card transactional database in \subjectname. 192 million transactions from 1.3 million customers  will be collected from a 12 month period prior to the observation point (June 2014) in the experiment. 

Merchants are assigned what is known as a merchant category code (MCC), it is used generally for classifying the primary business of the merchant\footnote{\url{https://www.visa.com/supplierlocator/}}. These codes have been leveraged in \subjectname\ to create a Money Manager Application which rolls these MCC in MCC categories. This application allows personal customers to keep track of their spending behaviour through a combination of reports, visualisations and search functionality. These categories are broken down into parent and child categories for this application. These categories can be seen in Table \ref{table:mcc} below

\begin{table}[H]
	\centering
	\resizebox{\textwidth}{!}
	{
		\label{my-label}
		\begin{tabular}{|l|l|l|l|}
			\hline
			\textbf{MCC Category} & \textbf{Parent}          & \textbf{Child}                 & \textbf{Spend / Live} \\ \hline
			2.2                   & Bills \& Utilities       & Cable/Satellite TV \& Internet & Spend                 \\ \hline
			2.7                   & Bills \& Utilities       & Gas/Electricity/Energy         & Live                  \\ \hline
			3.1                   & Leisure \& Entertainment & Cinema \& Theatre              & Spend                 \\ \hline
			4.1                   & Shopping                 & Groceries                     & Live                  \\ \hline
			4.4                   & Shopping                 & Clothing \& Accessories        & Spend                 \\ \hline
			5.2                   & Health \& Personal Care  & Doctor                         & Live                  \\ \hline
			5.6                   & Health \& Personal Care  & Hair \& Beauty                 & Spend                 \\ \hline
			6.2                   & Household/Home           & Household Maintenance          & Live                  \\ \hline
			6.5                   & Household/Home           & Computers \& Technology        & Spend                 \\ \hline
		\end{tabular}
	}
	\caption{MCC Categories and Spend/Live Categorisation Sample}
	\label{table:mcc}
\end{table}

For this research and experiment these transactions MCC categories have been assigned \textit{Spend} or \textit{Live} category. Transactions categorised as \textit{Spend} will translate to what has been derived as discretionary spend of \subjectname\ customers. This will include transactions such as social activities such as going to the cinema, going to bars and clubs and eating out in restaurants which can be seen in Table \ref{table:mcc}. Transactions categorised as \textit{Live} will translate to what as been derived as transactions required/associated/needed for customers to live. This will include transactions such as paying bills, shopping/groceries, healthcare etc.

Transactions for each customer address will be aggregated to electoral division and local authority for each month for the 12 months prior to the observation point. Metrics will be created calculating the difference between time periods to evaluate what trends or patterns are useful for identifying default. 

The rates of change between transactional spend will be calculated by using the percentage change equation seen below

\begin{equation} \label{eqn:PercentageChange}
\text{Percentage Change} = \frac{X^{2} - X^{1}}{X^{1}}*100  
\end{equation}
Where: 
\vspace{-7mm} 
\begin{itemize}
	\item $X^{1} = $ the original variable
	\item $X^{2} = $ the new variable
\end{itemize}

These features will not be normalised so feature rescaling will be applied to ensure the features range value is between 0 and 1, this is down through applying the rescaling equation found below  

\begin{equation}\label{eqn:rescaling}
\text{Feature Scaling} =  X' = \frac{X - X_{min}}{X_{max} - X_{min}}
\end{equation}

There will be 68 features created as part of this experiment. The names, data types and description for each features can be found in Appendix \ref{AppendixA}.



\subsubsection{Personal Loans \& Homeloans default behaviour}
An experiment will be carried out as part of this research to investigate if \subjectname's loan and homeloan default ratios and trends could be useful in predicting if SME customers will default. The intuition for this insight is that if there is a high proportion of personal loan and homeloan defaulting in an area then there could be a relationship with SMEs also defaulting. It was also discussed in literature by \cite{di_pietro_regional} how metrics like this were useful for predicting default in Italy.

At the observation point (June 2014), personal loan and homeloan products that were active on the \subjectname\ book were analysed. Table \ref{table:AIB_Products} details the products that were extracted as part of this experiment.

\begin{table}[H]
	\centering
		\small
			\begin{tabular}{llll}
			\hline
			\textbf{Product Category} & \textbf{Product Name} \\ \hline
			     & Fixed Loans \\ 
			    & Matrix Loans \\ 
			     & Other Loans \\ 
			Branch Advances     & Premium Business Rate \\ 
			     & Prime Loans \\ 
			     & Staff Credit Flex \\ 
					& Suspense Interest \\ \hline
			           & Buy to Let \\ 
			           & Commercial Mortgages \\ 
			           & Home Loan \\ 
			Home Loan           & Staff Homeflex \\ 
			           & Standard Mortgages \\ 
			           & Surplus Builder \\ 	
			           & Tracker Mortgages \\ \hline		
		\end{tabular}
	\caption{\subjectname\ Personal Loan and Homeloan products}
	\label{table:AIB_Products}
\end{table}

Each product in the Table \ref{table:AIB_Products} above will be rolled up to its product category. Each product will be associated with a customer and a customer to an address so they can be aggregated to electoral division and local authority. The products are then classified as default or not default so a ratio between the two per can be built for each electoral division and local authority. Fig. \ref{fig:BothPersonalRatio} below illustrates the default rates for each local authority for personal loan and homeloan products combined. 

\begin{figure}[H]
	\includegraphics[width=1\textwidth,center]{BothPersonalRatio}
	\caption{Default Ratios for Homeloan and Personal Loans by Local Authority}
	\label{fig:BothPersonalRatio}
\end{figure}

There will be 6 features created as part of this experiment. The names, data types and description for each features can be below in Table \ref{table:arrears_ratio} .

\begin{table}[H]
	\centering
	\small
	\resizebox{\textwidth}{!}
	{
		\begin{tabular}{|l|l|l|}
			\hline
			\textbf{Feature} & \textbf{Data Type} & \textbf{Description}                                                                                                                                                                                                                                                                                                                                                                                                                                                                                                                              \\ \hline
			ED\_HOME\_RATIO  & Interval & Homeloan Default Ratio for each Electoral Division \\ \hline
			ED\_LOAN\_RATIO  & Interval & Personal Loan Default Ratio for each Electoral Division \\ \hline
			ED\_TOTAL\_RATIO & Interval & Homeloan \& Personal Loan Default Ratio for each Electoral Division \\ \hline
			LA\_HOME\_RATIO  & Interval & Homeloan Default Ratio for each Local Authority \\ \hline
			LA\_LOAN\_RATIO  & Interval & Personal Loan Default Ratio for each Local Authority \\ \hline
			LA\_TOTAL\_RATIO & Interval & Homeloan \& Personal Loan Default Ratio for each Local Authority \\ \hline
		\end{tabular}
	}
	\caption{Personal Customers Arrears Ratio features derived for experiment}
	\label{table:arrears_ratio}
\end{table}

\subsubsection{Central Statistics Office (CSO) Features}
An experiment will be carried out as part of this research to investigate if data collected as part of the Irish census is indicative of SME default. The Census is carried out every 5 years in the Republic of Ireland to paint a picture of living and social conditions of the population. It provides details to the smallest geographic areas which can be used for decision making and planning. For example to find what infrastructure and services need to be invested in each areas for example schools, job training centre and health care services\footnote{\url{http://census.ie/}}. 

The data from the census can be organised into 11 themes which can be illustrated below in Table \ref{table:censusThemes}

\begin{table}[H]
	\centering
	\small
	\begin{tabular}{|l|l|}
		\hline
		\textbf{Theme Number} & \textbf{Theme Description} \\ \hline
		Theme 1    & Sex, age and marital status \\ \hline
		Theme 2    & Migration, ethnicity and religion \\ \hline
		Theme 3    & Irish Language \\ \hline
		Theme 4    & Families \\ \hline
		Theme 5    & Private Household \\ \hline
		Theme 6    & Housing \\ \hline
		Theme 7	   & Communal establishments \\ \hline
		Theme 8    & Principal status \\ \hline
		Theme 9    & Social class and socio-economic group \\ \hline
		Theme 10   & Education \\ \hline
		Theme 11   & Commuting \\ \hline	
		Theme 12   & Disability, careers and general health \\ \hline		
		Theme 13   & Occupation \\ \hline
		Theme 14   & Industries \\ \hline
		Theme 15   & PC and internet Access \\ \hline			
	\end{tabular}
	\caption{Themes of data available in the Irish census 2011}
	\label{table:censusThemes}
\end{table}


Anecdotally low unemployment rates and higher education rates would be associated with areas good economic growth and prosperity. Since the financial crisis of 2008-2009 there has been a huge reduction in the number of construction projects which is said to be heavily linked to higher unemployment rates. As a result for this experiment features have been created using data provided from the census at local authority and electoral division areas. The features that have been created are lower than secondary level of eduction ratio, manual based occupation ratio, and unemployment ratio by electoral division and local authority.

There will be 6 features created as part of this experiment. The names, data types and description for each features can be found below in Table \ref{table:censusFeatures}.


\begin{table}[H]
	\centering
	\resizebox{\textwidth}{!}
	{
		\label{my-label}
		\begin{tabular}{|l|l|l|}
			\hline
			\textbf{Feature} & \textbf{Data Type} & \textbf{Description}                                                                                                                                                                                                                                                                                                                                                                                                                                                                                                                              \\ \hline
			ED\_UNEMPLOYMENT\_RATIO  & Interval & Employment Ratio for each Electoral Division \\ \hline
			ED\_NON\_MANUAL OCCUPATION\_RATIO  & Interval & Non Manual Occupation Ratio for each Electoral Division \\ \hline
			ED\_LOWER\_THAN\_UPPER\_SECONDARY\_RATIO  & Interval &  Lower than Secondary Level/Leaving Certificate Ratio for each Electoral Division \\ \hline
			LA\_UNEMPLOYMENT\_RATIO  & Interval & Employment Ratio for each Local Authority \\ \hline
			LA\_NON\_MANUAL\_OCCUPATION\_RATIO  & Interval & Non Manual Occupation Ratio for each Local Authority \\ \hline
			LA\_LOWER\_THAN\_UPPER\_SECONDARY\_RATIO  & Interval & Lower than Secondary Level/Leaving Certificate Ratio for each Local Authority \\ \hline 
		\end{tabular}
	}
	\caption{(CSO) Features derived for experiment }
	\label{table:censusFeatures}
\end{table}


\subsubsection{SME Default behaviour}
An experiment will be carried out as part of this research to investigate if SME default ratios and behavioural changes could be useful in predicting future SME defaults. At an anecdotal level you could think of this as domino effect where businesses near each other close in sequence due to economic hardship in that area and low consumer confidence. For example one feature could be have default rates increased or decreased for SME customers at an electoral division or local authority area. It was also discussed in literature by \cite{di_pietro_regional} how this was useful for predicting default in Italy. 

The data for this experiment was extracted from the SME database in \subjectname\ from June 2012 to the June 2014(Observation Point). Each SME customers local authority and electoral division was identified, and customers were classified as in default or not in default for each month in the two year period. Features were then built with a focus of identifying local authorities and electoral division which had seen higher default rates in the two years prior to the observation point. 

The majority of the features will be built using the percentage change formula seen above in Equation \ref{eqn:PercentageChange} and the rescaling formula Equation \ref{eqn:rescaling}. But it will also include the ratio between default and non-default, and count of default for each local authority and electoral division.

There will be 20 features created as part of this experiment. The names, data types and description for each features can be found below in Table \ref{table:SMEDefaultBehaviourDataset}.

\begin{table}[H]
	\centering
	\resizebox{\textwidth}{!}
	{
		\begin{tabular}{|l|l|l|}
			\hline
			\textbf{Feature} & \textbf{Data Type} & \textbf{Description}                                                                                                                                                                                                                                                                                                                                                                                                                                                                                                                              \\ \hline
			LA\_PERCENT\_30\_06\_2014 & Interval & SME Default Ratio for each local authority at June 2014 \\ \hline
			ED\_PERCENT\_30\_06\_2014 & Interval & SME Default Ratio for each electoral division at June 2014 \\ \hline
			LA\_CNT\_1\_30\_06\_2014 & Interval & \# SME Defaults for each local authority at June 2014 \\ \hline
			ED\_CNT\_1\_30\_06\_2014 & Interval & \# SME Defaults for each electoral division at June 2014\\ \hline
			DIFF\_12\_2013\_LA & Interval & Difference between number of SME defaults from June 2014 to December 2013 for each local authority \\ \hline
			DIFF\_06\_2013\_LA & Interval & Difference between number of SME defaults from June 2014 to June 2013 for each local authority \\ \hline
			DIFF\_12\_2012\_LA & Interval & Difference between number of SME defaults from June 2014 to December 2012 for each local authority \\ \hline
			DIFF\_06\_2012\_LA & Interval & Difference between number of SME defaults from June 2014 to June 2012 for each local authority \\ \hline
			
			DIFF\_12\_2013\_ED & Interval & Difference between number of SME defaults from June 2014 to December 2013 for each electoral division \\ \hline
			DIFF\_06\_2013\_ED & Interval & Difference between number of SME defaults from June 2014 to June 2013 for each electoral division \\ \hline
			DIFF\_12\_2012\_ED & Interval & Difference between number of SME defaults from June 2014 to December 2012 for each electoral division \\ \hline
			DIFF\_06\_2012\_ED & Interval & Difference between number of SME defaults from June 2014 to June 2012 for each electoral division \\ \hline
			
			DIFF\_PERCENT\_12\_2013\_LA & Interval & Difference between SME defaults rates from June 2014 and December 2013 by each local authority \\ \hline
			DIFF\_PERCENT\_06\_2013\_LA & Interval & Difference between SME defaults rates from June 2014 and June 2013 by each local authority \\ \hline
			DIFF\_PERCENT\_12\_2012\_LA & Interval & Difference between SME defaults rates from June 2014 and December 2012 by each local authority \\ \hline
			DIFF\_PERCENT\_06\_2012\_LA & Interval & Difference between SME defaults rates from June 2014 and June 2012 by each local authority \\ \hline
			
			DIFF\_PERCENT\_12\_2013\_ED & Interval & Difference between SME defaults rates from June 2014 and December 2013 by each electoral division \\ \hline
			DIFF\_PERCENT\_06\_2013\_ED & Interval & Difference between SME defaults rates from June 2014 and June 2013 by each electoral division \\ \hline
			DIFF\_PERCENT\_12\_2012\_ED & Interval & Difference between SME defaults rates from June 2014 and December 2012 by each electoral division \\ \hline
			DIFF\_PERCENT\_06\_2012\_ED & Interval & Difference between SME defaults rates from June 2014 and June 2012 by each electoral division \\ \hline
		\end{tabular}
	}
	\caption{SME default behaviour features derived for experiment}
	\label{table:SMEDefaultBehaviourDataset}
\end{table}

\subsubsection{Feature grouping based on Home Loans, Personal Loans, SME Loans}
As reviewed in the literature in Section \ref{sec:binning}, binning is a useful method of transforming interval based features into categorical features. This has a number of benefits such as simplifying the structure of the data into nominal and ordinal based features and interval based on estimating the groups weight of evidence (WoE). Binning increases scoring models robustness by generalising model for unseen data thus reducing the chances of over-fitting. As discussed in the literature it also has the capability to incorporate missing values and other extreme outliers which cause instability in the model. 

The features to be binned as part of this experiment are based on features already seen in this section. Homeloan, personal loan and SME default ratios at the observation point (June 2014) will be binned in attempt to build robust and simplified feature set. This experiment will test the nominal/ordinal outputs and WoE outputs in the predictive modelling stage to see if either improve the prediction of the baseline model.

There will be 16 features created as part of this experiment. The names, data types and description for each features can be found below in Table \ref{table:grouped_features}.

\begin{table}[H]
	\centering
	\resizebox{\textwidth}{!}
	{
	\begin{tabular}{|l|l|l|}
		\hline
		\textbf{Feature} & \textbf{Data Type} & \textbf{Description}                                                                                                                                                                                                                                                                                                                                                                                                                                                                                                                              \\ \hline			
		GROUPED\_ED\_ID\_SME\_ARREARS & Ordinal/Nominal & Group/Bin feature based on information gain for electoral division default ratios for SME loans\\ \hline
		GROUPED\_LA\_ID\_SME\_ARREARS & Ordinal/Nominal & Group/Bin feature based on information gain for local authority default ratios for SME loans   \\ \hline
		WEIGHT\_OF\_EVIDENCE\_ED\_ID\_SME\_ARREARS & Interval & WoE for each bin value based on electoral division default ratios for SME loans\\ \hline
		WEIGHT\_OF\_EVIDENCE\_LA\_ID\_SME\_ARREARS & Interval & WoE for each bin value based on local authorities default ratios for SME loans \\ \hline
		
		GROUPED\_LA\_ID\_BOTH\_ARREARS & Ordinal/Nominal & Group/Bin feature based on information gain for local authority default ratios for Homeloan \& Personal loan\\ \hline
		WEIGHT\_OF\_EVIDENCE\_LA\_ID\_BOTH\_ARREARS & Interval & WoE for each bin value based on local authorities default ratios for Homeloans and Personal Loans \\ \hline
		
		GROUPED\_ED\_ID\_BOTH\_ARREARS & Ordinal/Nominal & Group/Bin feature based on information gain for electoral division default ratios for Homeloans \& Personal loans\\ \hline
		WEIGHT\_OF\_EVIDENCE\_ED\_ID\_BOTH\_ARREARS & Interval & WoE for each bin value based on electoral division default ratios for Homeloans and Personal Loans \\ \hline
		
		GROUPED\_ED\_ID\_BRANCH\_ADVANCE\_ARREARS & Ordinal/Nominal & Group/Bin feature based on information gain for electoral division default ratios for Personal loan only \\ \hline
		WEIGHT\_OF\_EVIDENCE\_ED\_ID\_BRANCH\_ADVANCE\_ARREARS & Interval & WoE for each bin value based on local authorities default ratios for Personal Loans \\ \hline
		
		GROUPED\_LA\_ID\_BRANCH\_ADVANCE\_ARREARS & Ordinal/Nominal &  Group/Bin feature based on information gain for local authorities default ratios for Personal loans only \\ \hline
		WEIGHT\_OF\_EVIDENCE\_LA\_ID\_BRANCH\_ADVANCE\_ARREARS & Interval & WoE for each bin value based on electoral division default ratios for Personal Loans \\ \hline
		
		GROUPED\_ED\_ID\_HOMELOANS\_ARREARS & Ordinal/Nominal & Something \\ \hline
		WEIGHT\_OF\_EVIDENCE\_ED\_ID\_HOMELOANS\_ARREARS & Interval & WoE for each bin value based on electoral divisions default ratios for Homeloans \\ \hline
		
		GROUPED\_LA\_ID\_HOMELOANS\_ARREARS & Ordinal/Nominal & Group/Bin feature based on information gain for local authorities default ratios for Homeloans only  \\ \hline
		WEIGHT\_OF\_EVIDENCE\_LA\_ID\_HOMELOANS\_ARREARS & Interval & WoE for each bin value based on local authorities default ratios for Homeloans \\ \hline
		
	\end{tabular}
	}
	\caption{Binned/Grouped features based on default ratios of SME, Home and Personal Loans by Electoral Division and Local Authority}
	\label{table:grouped_features}
\end{table}

\section{Generating the ABTs}
The data that will be collected and processed as part of this research has been outlined in the previous sections. It can be seen in Fig. \ref{fig:datamanipulation} how each internal data source in \subjectname\ will have to mapped to an address through the address matching application. 

\begin{figure}[H]
	\includegraphics[width=1\textwidth, height = 12cm,center]{datamanipulation}
	\caption{ABT generation process}
	\label{fig:datamanipulation}
\end{figure}

When the process is finished there will be two ABTs generated for SME customers based on features from historic scorecards and two ABTs that also incorporate experimental features based on the electoral division and local authority which will be investigated as part of this research.

\begin{table}[H]
	\centering\
	\resizebox{\textwidth}{!}
	{
		\begin{tabular}{l r r r r r r}
			\hline
			\textbf{Model} &  \textbf{\# Numeric} & \textbf{\# Nominal} & \textbf{\# Observations} & \textbf{\# Good} & \textbf{\# Bad} & \textbf{Good:Bad}\\
			\hline
			Previous Delinquency & 119(Previously 11) & 8(Previously 0) & 2,926 & 2,198  & 738 & 75:25 \\ 
			No Previous Delinquency & 117(Previously 9)  & 8(Previously 0) & 24,156 & 23,505 & 651  & 97:03 \\ \hline
			\textbf{Total} &  &  & \textbf{27,082} & \textbf{25,703} & \textbf{1,389} & \textbf{95:05} \\ \hline
		\end{tabular}
	}
	\caption{Characteristics of datasets to be used in the experiment \\to evaluate macro-economic features \\
		\# Numeric refers to the number of continuous features \\
		\# Nominal refers to the number of categorical features
	}
	\label{characteristicsMacroDatasets}
\end{table}

\section{Software Used}\label{chp3:softwareUsed}
A number of software tools and applications were required in this research. These have been broken down into four main subsections; address matching, data wrangling to create ABT, Visualisations and Modelling. All the software used in this research was open source apart from SAS which is used to build the for the benchmark evaluation experiment in Section \ref{sec:benchFeature}. 
 
\subsubsection{Address Matching}
A combination of Apache Solr\footnote{\url{http://lucene.apache.org/solr/}} and Python\footnote{\url{https://www.python.org/}} were used to match \subjectname's addresses to an ED/LA in GDD. Solr ia an open source web application enterprise search engine. It is an open source application written in Java, which is a wrapper around Apache Lucene\footnote{{\url{https://lucene.apache.org/core/}}}. Combined they provide a reliable, fast, scalable platform capable of providing distributed indexing which can then be used for searching or navigation. Solr was used to index GeoDirectory Database which then allows it to be searched. Python is a high-level, general purpose programming language which can be used to build both large and small scale programs. Python like Solr is open source and freely available. One its most attractive and best characteristics is it is easy to read and use.

\subsubsection{Data Wrangling to create ABT}
Anecdotally speaking, data scientists and analysts spend majority of their time data wrangling. Data wrangling is time consuming mundane process used to collect and prepare data prior to being explored for useful information. In the experiment of this thesis a variety of data types and data sources were used. Data from the \subjectname\ EDW, GDD in Solr served in JSON, CSO data in flat file to name a few. R\footnote{{\url{https://www.r-project.org/}}}, an open source programming language has much more emphasis on statistical computing. It also has many libraries available for processing and data manipulation which are available in the CRAN\footnote{{\url{https://cran.r-project.org/}}} repository. The most useful package used during this process was \textit{reshape}\footnote{{\url{https://cran.r-project.org/web/packages/reshape/index.html}}}. Reshape allows to easily restructure, transpose and aggregate data.

\subsubsection{Visualisations}
R is also a very strong programming language at creating beautiful visualisations so it will be used throughout this paper. One package used in this paper was \textit{ggplot}
\footnote{{\url{https://cran.r-project.org/web/packages/ggplot2/index.html}}}. 

For Geographic Information Systems (GIS) that have been created as part of this research QGIS\footnote{\url{http://www.qgis.org/en/site/}} has been used. QGIS formally known as Quantum GIS is a cross platform open source GIS application allowing you to create maps compiled with data for information sharing and analysis. It is the challenger to the proprietary application ArcGIS\footnote{\url{https://www.arcgis.com/features/}} both provide a very similar interface and functionality for creating and sharing maps.

\subsubsection{Modelling}
SQL was used to identify customers to be used for predictions and generate the target class.
The models and experiments performed in this research were built in R and SAS. SAS is a proprietary software. SAS is the tool of choice by the modelling teams in \subjectname. Anecdotally SAS is a legacy system in financial institutions, it is what people are used to using but also there is a for-profit corporation vetting the code for its customers, offers customer service and support which corporations are used to. SAS offers a graphical interfaces which means users do not have to enter code, but this can be complemented using the SAS programming language. Anecdotally speaking SAS is excellent for building predictive modes resulting in good time to value. SAS Enterprise Miner includes the following components Time Series, Variable Clustering, Cluster, Interactive Binning, Principal Components, AutoNeural, DMNeural, DMine Regression, Gradient Boosting, Ensemble, and Text Mining.
\\
R is a very strong competitor to SAS in this space. Because of its open source nature there are many libraries available for building predictive models. For example one popular package \textit{caret}\footnote{{\url{https://cran.r-project.org/web/packages/caret/index.html}}} contains many models and continues to grow.

The experiment in Section \ref{sec:benchFeature} will built using SAS. Experiments in later section will be completed using R. This was done as to leverage the industry supported tool for decisions such as model selection and performance measure selection. For further experiments it was found that R had a much quicker time to value for manipulation of data, feature selection and model comparison. For R the data was being stored locally on the authors machine in comparison to SAS which runs on a industry server in \subjectname\ which is heavily administrated.  


\section{Conclusions}\label{desConc}
This chapter has discussed what data will be used as part of this research project. 

The chapter is broken into two categories, the first details what small-medium enterprises(SME/SMEs) customers the prediction model will be trained on, and the time-period the prediction will be made on. It is discussed how a benchmark model will be built using features selected from a historic \subjectname\ credit scorecard which will be used for comparisons carried out as part of the research experiment.

The second category discussed the macro-economic features that were generated as part of this research project. These features were generated from internal sources in \subjectname\ and open data from the Census. An address matching application was designed and built as part of the macro-economic feature generation stage which mapped customer address data on \subjectname\ to geographic regions in Ireland. \subjectname\ were so impressed by the results that a project to evaluate the application will be carried to see if it can be leveraged for matching addresses in \subjectname\ to the Eircode system.

A methodology for generating multiple analytical base tables is detailed which covers where the data is sourced and where data wrangling is carried out. 

Finally software that was used in the data generation stage is discussed.
