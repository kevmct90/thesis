% Chapter Template

\chapter{Design \& Methodology} % Main chapter title

\label{Chapter4} % Change X to a consecutive number; for referencing this chapter elsewhere, use \ref{ChapterX}

\lhead{Chapter 4. \emph{Design \& Methodology}} % Change X to a consecutive number; this is for the header on each page - perhaps a shortened title

\section{Introduction}

Make sure say you are using the same data split for training and testing so results are consistent.
\textit{
As the standard approach to assess credit scoring
systems is to use a holdout test set, each dataset used was divided into three subsets:
(i) the training set (55\%); (ii) the validation set (15\%), and (iii) the test set (30\%).
The training set and the validation set were used to train and tune the classifiers
while the test set was used to assess their performance. This procedure was performed
repeatedly over a number of turns \citep{kennedy_credit_2013}}