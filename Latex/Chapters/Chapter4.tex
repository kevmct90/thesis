% Chapter Template

\chapter{Experiment Implementation and Evaluation} % Main chapter title

\label{Chapter4} % Change X to a consecutive number; for referencing this chapter elsewhere, use \ref{ChapterX}

\lhead{Chapter 4. \emph{Implementation / results evaluation}} % Change X to a consecutive number; this is for the header on each page - perhaps a shortened title

%----------------------------------------------------------------------------------------
%	SECTION 1
%----------------------------------------------------------------------------------------
\section{Introduction}
This chapter will present the implementation of experiments performed in the attempt to predict drop-offs from the loan application process on \subjectname's internet banking website. The predictive capacity of the standard aggregate summaries of click patterns will be discussed as well as evaluating the usefulness of the ordinal rank encoding applied to events in the loan application process.


%----------------------------------------------------------------------------------------
%	SECTION 2
%----------------------------------------------------------------------------------------
\section{Baseline Benchmarks in \subjectname\ }
When building predictive models, it is useful to have a benchmark with which to form a basis for model comparison that is not simply the \textit{no information rate}. The model will ultimately be deployed for decision making in real time so models that are overly complex will need to compensate for additional complexity with substantial up-lift in accuracy. 


\section{Feature Selection}
With respect to creating a model that could be utilised in real time, reduction of dimensionality is of key importance. Generating all of the 400 metrics used in the initial model based on on-going session activity would be very complex to implement in practice. A feature selection process was implemented at this point to reduce the complexity of the dataset. 
\subsection{Correlation}
\subsection{Stepwise Analysis}

\section{Binning / Coarse Classification}

\section{Addressing Imbalance}
\subsection{Undersampling}
\subsection{Oversampling}

\section{Comparing Results of Various Experiments}

\section{Model Selection/Validation}

\section{Interpretation of Results}

\section{Evaluation of Experiments}

\section{Conclusion}