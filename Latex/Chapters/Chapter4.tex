% Chapter Template

\chapter{Design \& Methodology} % Main chapter title

\label{Chapter4} % Change X to a consecutive number; for referencing this chapter elsewhere, use \ref{ChapterX}

\lhead{Chapter 4. \emph{Design \& Methodology}} % Change X to a consecutive number; this is for the header on each page - perhaps a shortened title

\section{Introduction}

This chapter will present the design of the experiment to be undertaken as part of this research. The aim of this research is see if including macro-economic features while training a SME credit scoring model will it return better results than a model just trained using the benchmark features from a historic scorecard. 

There will be two main sections to this chapter. The first we will be building and evaluating the benchmark features for \textit{previous delinquency} and \textit{no previous delinquency} models. This will involve evaluating the predictive capabilities of various models discussed in Chapter \ref{Chapter2}.



\section{Baseline Benchmark Evaluation}

\begin{figure}[H]
	\includegraphics[width=1\textwidth,height = 11cm,center]{exploratoryevaluationmodel}
	\caption{Default Ratios for Homeloan and Personal Loans by Local Authority}
	\label{fig:exploratoryevaluationmodel}
\end{figure}





\section{Baseline Benchmark Evaluation}

\begin{figure}[H]
	\includegraphics[width=1\textwidth,height = 11cm,center]{Experiment_Workflow}
	\caption{Default Ratios for Homeloan and Personal Loans by Local Authority}
	\label{fig:Experiment_Workflow}
\end{figure}


\pagebreak


\section{Old Sections}
This chapter will detail the design of the experiments that are to be implemented as part of the research project. It will detail how customers addresses are mapped to locations and how features are generated for these locations. It will also discuss any of the pre-processing, and feature selection and statistical techniques used to create select the data for the ABT.

The SME default problem would be considered imbalanced dataset for classification problem where algorithms make the assumption there is a equal number of good and bad classes/group \citep{japkowicz_class_2000}. The dataset used in this experiment has a small number of SME customer in default, because of this issue a number of approaches will taken to considered to address.

Statistical analysis and evaluation methods used to assess the strength of models will be detailed and discussed, also it will outline any further transformations necessary for specific algorithms. Finally criteria for deciding which model will be chosen as the best and most fit for purpose will be detailed.


\subsection{Generating the ABTs}
We now have a reference list that allows us to apply a rank to all achievable loan application events prior to completion. A method is required to join this reference list with our csv (comma seperated value) log files resulting from the MapReduce job. 

%Fine from here
\subsection{Software Used}\label{softwareUsed}

\subsubsection{Address Matching}
A combination of Apache Solr\footnote{\url{http://lucene.apache.org/solr/}} and Python\footnote{\url{https://www.python.org/}} were used to match \subjectname's addresses to an ED/LA in GDD. Solr ia an open source web application enterprise search engine. It is an open source application written in Java, which is a wrapper around the Apache Lucence \footnote{{\url{https://lucene.apache.org/core/}}}. Combined they provide a reliable, fast, scalable platform capable of providing distributed indexing which can then be used for searching or navigation. Solr was used to index GD which then allowed it to be searched. Python is a high-level, general purpose programming language which can be used to build both large and small scale programs. Python like Solr is open source and freely available. One its most attractive and best characteristics is it is easy to read and use. A program was created to standardise and cleanse the addresses in \subjectname's database. The program would then take the cleansed addresses and query them against GDD indexed through the Solr web API, using string comparison algorithms discussed in Chapter 2 to return the most likely GDD address.

\subsubsection{Data Wrangling}
Anecdotally speaking, data scientists and analysts spend majority of their time data wrangling. Data wrangling is time consuming mundane process used to collect and prepare data prior to being explored for useful information. In the experiment of this thesis a variety of data types and data sources were used. Data from the \subjectname\ EDW, GDD in Solr served in JSON, CSO data in flat file to name a few. R\footnote{{\url{https://www.r-project.org/}}}, another open source programming language but has much more emphasis on statistical computing. It also has many libraries available for processing and data manipulation which are available in the CRAN\footnote{{\url{https://cran.r-project.org/}}} repository. The most useful package used during this process was \textit{reshape}\footnote{{\url{https://cran.r-project.org/web/packages/reshape/index.html}}}. Reshape allows to easily restructure, transpose and aggregate your data. 

\subsubsection{Visualisations}
R is also a very strong programming language at creating beautiful visualisations so it will be used throughout this paper. One package used in this paper was \textit{ggplot}
\footnote{{\url{https://cran.r-project.org/web/packages/ggplot2/index.html}}}. 

Some other custom geospatial visualisations may also be required. This could be done using Arcgis, Qgis, D3 or R still to be decided.

\subsubsection{Modelling}
SQL was used to identify customers to be used for predictions and generate the target class.
The models and experiments performed in this used in this are built in R and SAS. SAS is a proprietary software. SAS is the tool of choice by the modelling teams in \subjectname. Anecdotally SAS is a legacy in financial institutions, it is what people are used to using but also there is a for-profit corporation vetting the code for its customers and customer service and support corporations are used to. SAS offers a graphical interfaces which means users do not have to enter code, but this can be complemented using the SAS programming language. Anecdotally speaking SAS is excellent for building predictive modes resulting in good time to value. SAS Enterprise Miner includes the following components Time Series, Variable Clustering, Cluster, Interactive Binning, Principal Components, AutoNeural, DMNeural, DMine Regression, Gradient Boosting, Ensemble, and Text Mining.
\\
R is a very strong competitor to SAS in this space. Because of its open source nature there are many libraries available for building predictive models. For example one popular package \textit{caret}\footnote{{\url{https://cran.r-project.org/web/packages/caret/index.html}}} contains many models and continues to grow

\section{Building Models}
In order to avoid over-fitting models, considerations need to be made as to what data to use in training the model. Each of the models built will be trained using a training set which is a subset of the over all available samples. We discussed possible options for training set selection in subsection


\section{Evaluation Methods}
Several evaluation techniques will be applied to quantify the usefulness of all models produced. As discussed in section \ref{modelEval}, there are various ways to evaluate the performance of a classifier. A common metric used is overall accuracy (\% of records correctly classified). 

