% Chapter Template

\chapter{Experiment Implementation \& Evaluation} % Main chapter title

\label{Chapter4} % Change X to a consecutive number; for referencing this chapter elsewhere, use \ref{ChapterX}

\lhead{Chapter 4. \emph{Experiment Implementation \& Evaluation}} % Change X to a consecutive number; this is for the header on each page - perhaps a shortened title

%----------------------------------------------------------------------------------------
%	SECTION 1
%----------------------------------------------------------------------------------------
\section{Introduction}
This chapter describes the experiment set-up and methodology for this project to predict if SME customers will go into arrears. The predictive capability of many macro location-based features will be discussed, examined and evaluated.

There is relatively limited literature to support the location-based features chosen in this thesis, decisions are largely made due to anecdotal evidence and subject matter experts knowledge. To address this issue, empirical analysis of these features will be completed where one set of results will cascade new results and inspire further analysis. 

To understand if the location-based features in this analysis are useful, an initial model will be used as a baseline model from which we will look to improve iteratively. The features from this model will be derived from a historic scorecard used in \subjectname.  

%----------------------------------------------------------------------------------------
%	SECTION 2
%----------------------------------------------------------------------------------------
\section{Experiment Set-up}
This section will detail high level experimental information. In Fig. \ref{fig:experiment_setup} below SME accounts/customers are selected at the \textit{observation point}, June 2014. These accounts are not in default at the point. Information prior to the observation point will be used for modelling to predict if someone is likely to go into default, this can be termed the \textit{performance window}.

Data from each individual customers/accounts performance will be taken from data in the performance window time period, which will be combined with macro location-based data prior to the observation point also. This data will be aggregated and structured into features for an ABT. The aim will be that these features will be able to distinguish what customers/accounts are likely to default on there repayment in the next 12 months. 

\begin{figure}[h!]
	\includegraphics[width=0.8\textwidth,center]{experiment_setup}
	\caption[Experiment Performance Window and Outcome Window]
	{Experiment Performance Window and Outcome Window}
	\label{fig:experiment_setup}
\end{figure}

Previous, Section \ref{classLabelDef} described that there were two methods used to define if a customer was in default or not: (i) the \textit{worst status} label definition method; and (ii) the \textit{current status} label method approach. As mentioned previously, for the purpose of this experiment we will be using the industry standard worst status method

\section{Baseline Benchmark in \subjectname\ }
A model will be built as a baseline benchmark to compare the results of the models run in the experiments. The baseline model will actually be two models, based on a historic scorecards in \subjectname. One model will be based of features from a scorecard for customers who have been delinquent in the past and the other for customers who have not been in delinquent in the past. The customers will be modelled with these features first and results will be recorded. As part of the experiment Geo-location features will be added to be modelled, with the aim that these features will statistically significant for predicting SME arrears. 

As mentioned before at the observation point, June 2014 SME accounts were selected that were not in default. These accounts are the basis for this experiment with a total a population size of is 27,082. 


\begin{table}[H]
	\centering
	\begin{tabular}{l| p{5cm}| r}
		\hline
		\textbf{Scorecard} & \textbf{Status After Outcome Window June 2015} & \textbf{\# SME Accounts} \\
		\hline
		Previous Delinquency          & In Default        & 728 \\
		Previous Delinquency          & Not in Default        & 2,198 \\ 
		No Previous Delinquency          & In Default        & 651 \\ 
		No Previous Delinquency          & Not In Default        & 23,505 \\
		\hline
	\end{tabular}
	\caption{Base Model Account Detail}
\end{table}

The industry model of choice for credit scoring is logistic regression so that will be used here for building the base line model also. The two segmented models are are ran separately, and are then combined to give an overall accuracy of the predictive model. 


\subsubsection{Previous Delinquency}

\begin{table}[H]
	\centering
	\begin{tabular}{r | c | l}
		\hline
		\textbf{Model} & \textbf{AUC} & \textbf{GINI} \\
		\hline
		Dmine Regression          & 0.663        & 0.325 \\
		Gradient Boosting          & 0.651        & 0.302 \\
		Regression Backstep          & 0.644        & 0.288 \\
		Regression Forward Step          & 0.644        & 0.288 \\
		Regression Both          & 0.644        & 0.288 \\
		Regression          & 0.643        & 0.286 \\
		SVM Radial Basis Function          & 0.612        & 0.283 \\
		Regression Polynomial          & 0.612        & 0.224 \\
		Decision Tree          & 0.611        & 0.222 \\
		SVM Sigmoid          & 0.525        & 0.049 \\
		\hline
	\end{tabular}
	\caption{Previous Delinquency Base Model Details}
\end{table}


\begin{figure}[H]
	\includegraphics[width=0.9\textwidth,center]{Delinq_Model_ROC}
	\caption{Previous Delinquency Model ROC Chart}
	\label{fig:Delinq_Model_ROC}
\end{figure}

\begin{figure}[H]
	\includegraphics[width=0.9\textwidth,center]{Delinq_Model_Lift}
	\caption{Previous Delinquency Model Lift Chart}
	\label{fig:Delinq_Model_Lift}
\end{figure}


Choosing an appropriate cut-off or threshold.

\begin{table}[H]
	\centering
	\begin{tabular}{r | c | l}
		\hline
		\textbf{Cut-off/Threshold } & \textbf{Method} & \textbf{TP} \\
		\hline
		Regression Backstep          & 0.694        & 0.387 \\
		Regression Forward Step          & 0.694        & 0.387 \\
		Regression Both          & 0.694        & 0.387 \\
		Regression          & 0.693        & 0.387 \\
		Gradient Boosting          & 0.667        & 0.334 \\
		Dmine Regression          & 0.649        & 0.299 \\
		SVM Radial Basis Function          & 0.549        & 0.099 \\
		Decision Tree          & 0.50        & 0 \\
		SVM Sigmoid         & 0.458        & -0.084 \\
		\hline
	\end{tabular}
	\caption{No Previous Delinquency Base Model Details}
\end{table}

\subsubsection{No Previous Delinquency}

\begin{table}[H]
	\centering
	\begin{tabular}{r | c | l}
		\hline
		\textbf{Model} & \textbf{AUC} & \textbf{GINI} \\
		\hline
		Regression Backstep          & 0.694        & 0.387 \\
		Regression Forward Step          & 0.694        & 0.387 \\
		Regression Both          & 0.694        & 0.387 \\
		Regression          & 0.693        & 0.387 \\
		Gradient Boosting          & 0.667        & 0.334 \\
		Dmine Regression          & 0.649        & 0.299 \\
		SVM Radial Basis Function          & 0.549        & 0.099 \\
		Decision Tree          & 0.50        & 0 \\
		SVM Sigmoid         & 0.458        & -0.084 \\
		\hline
	\end{tabular}
	\caption{No Previous Delinquency Base Model Details}
\end{table}

\begin{figure}[H]
	\includegraphics[width=0.9\textwidth,center]{NonDelinq_Model_ROC}
	\caption{No Previous Delinquency Model ROC Chart}
	\label{fig:NonDelinq_Model_ROC}
\end{figure}

\begin{figure}[H]
	\includegraphics[width=0.9\textwidth,center]{NonDelinq_Model_Lift}
	\caption{No Previous Delinquency Model Lift Chart}
	\label{fig:NonDelinq_Model_Lift}
\end{figure}

\section{Feature Selection}
With respect to creating a model that could be utilised in real time, reduction of dimensionality is of key importance. Generating all of the 400 metrics used in the initial model based on on-going session activity would be very complex to implement in practice. A feature selection process was implemented at this point to reduce the complexity of the dataset. 

\subsection{Correlation Analysis}

\begin{figure}[H]
	\centering
		\begin{subfigure}[b]{0.32\textwidth}
			\captionsetup{font=scriptsize}
			\includegraphics[width=\textwidth,height=2.75cm]{Grouped_Features_Correlation_Analysis_Subset}\caption{Grouped Features}\label{fig:groupedFeaturesCorrelation}
		\end{subfigure} 
		\begin{subfigure}[b]{0.32\textwidth}
			\captionsetup{font=scriptsize}
			\includegraphics[width=\textwidth,height=2.75cm]{SME_Arrears_Trends_Correlation_Analysis_Subset}
			\caption{SME Arrears Trends}\label{fig:smeArrearsCorrelation}
		\end{subfigure} 
		\begin{subfigure}[b]{0.32\textwidth}
			\captionsetup{font=scriptsize}
			\includegraphics[width=\textwidth,height=2.75cm]{Personal_Arrears_Ratio}
			\caption{Personal Arrears}\label{fig:personalArrearsCorrelation}
		\end{subfigure} 
	\medskip
	\begin{subfigure}[b]{0.32\textwidth}
		\captionsetup{font=scriptsize}
		\includegraphics[width=\textwidth,height=2.75cm]{Visa_Debit_Correlation_Analysis_Subset}
		\caption{Transactions}\label{fig:transVisaCorrelation}
	\end{subfigure} ~\quad
	\begin{subfigure}[b]{0.32\textwidth}
		\captionsetup{font=scriptsize}
		\includegraphics[width=\textwidth,height=2.75cm]{CSO_Correlation_Analysis}
		\caption{CSO}\label{fig:CSOCorrelation}
	\end{subfigure}
\caption{Unbalanced Correlation Analysis}
\label{fig:unbal_corr_analysis}
\end{figure}
	
\begin{figure}[H]
	\centering
	\begin{subfigure}[b]{0.32\textwidth}
		\captionsetup{font=scriptsize}
		\includegraphics[width=\textwidth,height=2.75cm]{Balanced_Grouped_Features_Correlation_Analysis_Subset}\caption{Grouped Features}\label{fig:groupedFeaturesCorrelation}
	\end{subfigure} 
	\begin{subfigure}[b]{0.32\textwidth}
		\captionsetup{font=scriptsize}
		\includegraphics[width=\textwidth,height=2.75cm]{Balanced_SME_Arrears_Trends_Correlation_Analysis_Subset}
		\caption{SME Arrears Trends}\label{fig:smeArrearsCorrelation}
	\end{subfigure} 
	\begin{subfigure}[b]{0.32\textwidth}
		\captionsetup{font=scriptsize}
		\includegraphics[width=\textwidth,height=2.75cm]{Balanced_Personal_Arrears_Ratio}
		\caption{Personal Arrears}\label{fig:personalArrearsCorrelation}
	\end{subfigure} 
	\medskip
	\begin{subfigure}[b]{0.32\textwidth}
		\captionsetup{font=scriptsize}
		\includegraphics[width=\textwidth,height=2.75cm]{Balanced_Visa_Debit_Correlation_Analysis_Subset}
		\caption{Transactions}\label{fig:transVisaCorrelation}
	\end{subfigure} ~\quad
	\begin{subfigure}[b]{0.32\textwidth}
		\captionsetup{font=scriptsize}
		\includegraphics[width=\textwidth,height=2.75cm]{Balanced_CSO_Correlation_Analysis}
		\caption{CSO}\label{fig:CSOCorrelation}
	\end{subfigure}
	\caption{Balanced Correlation Analysis}
	\label{fig:balanced_corr_analysis}
\end{figure}


\subsection{Stepwise Analysis}

\section{Binning / Coarse Classification}

\section{Addressing Imbalance}
\subsection{Undersampling}
\subsection{Oversampling}

\section{Comparing Results of Various Experiments}

\section{Model Selection/Validation}

\section{Interpretation of Results}

\section{Evaluation of Experiments}

\section{Conclusion}