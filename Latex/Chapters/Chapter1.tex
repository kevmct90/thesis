% Chapter Template

\chapter{Introduction} % Main chapter title

\label{Chapter1} % Change X to a consecutive number; for referencing this chapter elsewhere, use \ref{ChapterX}

\lhead{Chapter 1. \emph{Introduction}} % Change X to a consecutive number; this is for the header on each page - perhaps a shortened title

% INTRO
In the financial market credit risk is one of the oldest forms of risk. If credit can be defined as "one party (lender) (he/she/it) trusting another party (borrower) so much that they are willing to provide them with money or a resource where they expect to be repaid not immediately but after some period of time", then credit risk the probability or chance that the borrower never repays the lender. Credit risk has existed since lending began itself dating back to 1800 B.C., where there has always been a certain amount of uncertainty in the lending borrowing process in the financial market \citep{caouette_managing_1998}.

There are many risks in the repayment process of lending where the lender will not receive the full payment, the principle, the interest or both. To help tackle this issue credit risk models have been deployed to identify what borrowers are likely to fail paying back their financial obligation \citep{sirirattanaphonkun_default_2012}

Two thirds of the labour workforce are employed by SMEs with less than 250 employees in countries in \textit{The Organisation for Economic Co-operation and Development} (OECD)\footnote{The OECD's origins date back to 1960, when 18 European countries plus the United States and Canada joined forces to create an organisation dedicated to economic development. Today, our 34 Member countries span the globe, from North and South America to Europe and Asia-Pacific. They include many of the world’s most advanced countries but also emerging countries like Mexico, Chile and Turkey. We also work closely with emerging economies like the People's Republic of China, India and Brazil and developing economies in Africa, Asia, Latin America and the Caribbean. Together, our goal continues to be to build a stronger, cleaner and fairer world. \url{http://www.oecd.org/about/membersandpartners/}} (\cite{beck_bank_2008}; \cite{dietrich_explaining_2012}).

\textit{There  are  numerous  studies  on  credit 
	risk based on financial data of listed companies, which are mostly 
	large  corporations,  but  very  few  studies  utilized  the  data  of  small 
	and  medium  enterprises  (SMEs).  This  could  be  mainly  due  to  the 
	concern of the reliability of the data since large firms listed at the 
	stock  exchanges  are  closely  monitored  by  the  authority  to  ensure 
	that the financial data provide accurate and useful information for 
	investors and shareholders. \citep{sirirattanaphonkun_default_2012}}

\textit{The
	implementation of reliable statistical methods to measure and forecast these probabilities implies the
	consideration of an observation period. In other words, the identification of default events implies mon
	-
	itoring each debtor over time and the identification of a transition from non-default to a default state.
	The obtained statistics are useful for credit institutions in many ways, spanning from the credit ap-
	proval process (for instance to implement cut-off points to screen credit applications), to the establish-
	ment of risk-sensitive pricing, the decision on whether collateral is required or not, the estimation of
	provisioning requirements or impairment losses, and the assessment of capital adequacy.
	\citep{antunes_estimating_2005}}

Studies demonstrate that performing SMEs are vital as they as seen as the spine of countries economies all around the world, because they are hubs for providing jobs, innovation and growth to the economy \citep{craig_sba-guaranteed_2004}.


\textit{For example, we know that during recessions, consumers
	are likely to cut back on luxuries, and thus firms in the consumer durable goods sector
	should see their credit risk increase. \citep{hackbarth_capital_2006}}

\textit{ \textbf{this paper will be useful for introduction motivation etc}
In the aftermath of the global financial crisis of
2008–2009, there has been an increased interest in the
role of small and medium enterprises in job creation
and economic growth. However the lack of consistent
indicators at the country level restricts extensive crosscountry
analyses of lending to small and medium
enterprises. This paper introduces a new dataset to fill this
gap in the small and medium enterprise data landscape.
In addition, it provides the first set of results of analyses
with this new dataset, predicting the global small and 
medium enterprise lending volume to be $10$ trillion.
The bulk of this volume, 70 percent, is in high-income
countries. On average, small and medium enterprise
loans constitute 13 percent of gross domestic product
in developed countries and 3 percent in developing
countries. Note that although a unique small and
medium enterprise definition does not exist, differences
in definitions across countries are not statistically
significant in explaining the differences in small and
medium enterprise lending volumes.
\citep{ardic_small_2011}
}

This chapter introduces the research topic, the predictiveness of location-based features in \textit{Small Medium Enterprises} SME credit scoring, and how it can potentially can be used to improve credit scoring in financial institutions. The motivation for the research question are discussed with details of its significance. The aims of the thesis along with the research objectives and analytical approach are then stated. The scope and limitation of this research are then described. The chapter concludes with a summary and outlines of this thesis.

\textit{
http://www.cso.ie/en/surveysandmethodology/multisectoral/businessdemography/
The CSO Business Demography results for 2012 found that 68% of Irish private sector employees work for an SME. They account for 52% of total employment.
}


\textit{footnotes found in \citep{rocha_status_2011} also very good introduction for references
According to Ayyagari et al., (2007) SMEs account for more than 60\% of manufacturing employment across 76
developed and developing economies. 
Schiffer and Weder (2001), IADB (2004) and Beck et al. (2005, 2006 and 2008) show SMEs perceive access to
finance and cost of credit to be greater obstacles than large firms and these factors affect their growth.
}


\cite{beck_bank_2008} research indicates that most commercial financial institutions consider the SME sector to be very profitable. 





%----------------------------------------------------------------------------------------
%	SECTION 1
%----------------------------------------------------------------------------------------

\section{Overview of Project}
Not sure I need this???

%----------------------------------------------------------------------------------------
%	SECTION 2
%----------------------------------------------------------------------------------------

\section{Background of the Problem}

We begin with a high-level overview of SME credit scoring and location data in machine learning.

\subsection{SME Credit Scoring}

Credit scoring, in layman's language is the name given to describe the process of assessing how risky customers and applicants are to default on their financial obligation \citep{hand_statistical_1997}. The goal is to split customers into two classes, 'good' and 'bad'. Customers in the good class are considered likely to repay on their financial obligation while customers in the bad class are likely to default on theirs. In order to make this decision about its customers, financial institutions use information known about the customer and other relevant metrics to identify how risky a customer is. 


Since the financial crisis of 2007-2008 there has been a  much greater emphasis on credit scoring in the consumer lending process in banking. One of the most common methods of building scorecards is by using data mining \citep{baesens_50_2009}. One of the benefits to improving the scoring accuracy of a credit model is obviously the significant future savings \citep{west_neural_2000} but also financial institutions are under increasing pressures from global (Bank for International Settlements) e.g. The European Central Bank and national bank e.g. Central bank of Ireland. Since the crisis these regulators police that financial institutions keep better care of their credit scoring systems. Poor performing credit systems can have massive adverse effect on that financial institutions profits, reputation and  ability to support the economy.
\\\\
There are three main stages of a typical credit scorecard development \cite{van_gestel_credit_2009}

\begin{itemize}
	\item Dataset Construction
	\item Modelling
	\item Documentation 
\end{itemize}

This can displayed using this architectural diagram. Insert find diagram.
\\\\
Each stage of this process cannot be left out when completing a industry credit scorecard, however for this research will be focusing particularly on the data construction and modelling stages. The documentation stage is critical for deployment of an industry scorecard but will not be investigated further in this thesis. 

%----------------------------------------------------------------------------------------
%	
%----------------------------------------------------------------------------------------

\section{Motivation}
Since the financial of 2007-2008 there is an even greater emphasis on quality credit scoring. In Ireland it was the property bubble that collapsed causing the banking crisis. 

"Many believe that focusing on lending on the SME sector could play an important role in restoring faith in the banking system and supporting economic growth, according to leading experts"


The primary aim of this research project is to build a quantitative credit scorecard that will identify Small Medium Enterprise (SME) customers who are likely to go into arrears/defaults.


The upheaval in the financial markets that accompanied the 2007-2008 sub-prime
mortgage crisis has emphasised the large proportion of the banking industry based
on consumer lending (Thomas, 2009b). Credit scoring is an important part of the
consumer lending process. It is an endeavour regarded as one of the most popular
application fields for both data mining and operational research techniques (Baesens
et al., 2009). Improving the scoring accuracy of the credit decision by as much as
a fraction of a percent can result in significant future savings (West, 2000). Furthermore,
global (Bank for International Settlements) and national (central banks)
regulators insist that financial institutions keep better track of their credit scoring
systems. The costs of incorrectly classifying a customer can be high, both financially
and in terms of reputation.

%----------------------------------------------------------------------------------------
%	SECTION 3
%----------------------------------------------------------------------------------------

\section{Research Project}
The research projects aims to build a SME credit scorecard, combine a base model in \subjectname\ combined with features based on SME locations. The aim of this scorecard is to identify SME customers who are likely to default and aim would be use these features in a scorecard in \subjectname.


%----------------------------------------------------------------------------------------
%	SECTION 4
%----------------------------------------------------------------------------------------

\section{Research Objectives}
The primary goals of this research is to investigate if location based metrics are indicative to whether a SME customer is likely to default on their financial obligation or not. The credit score model built will make use of a historic credit score model in \subjectname\ and this will be combined with location based metrics in an attempt to improve the scorecard. 
 
The objectives of this research are:

\begin{itemize}
	\item To study the relevant state-of-the-art literature and industry best practices on building credit scorecards.
	\item Create a search address engine and for mapping customer addresses to locations in an address database.
	\item Build location based metrics using internal data from \subjectname\ such as transactional, arrears ratio, post code and open data from the CSO.
	\item Identify most predictable features using feature reduction techniques.
	\item Build base model/scorecard to compare my research and experiment with.
	\item Build scorecard including location based metrics and evaluate results
	\item Assess is it possible to deploy the scorecard in \subjectname\ on out of time data.
	\item Assess whether using location based features have returns improved predictions compared to the existing models in \subjectname.
	\item Determine what future research could be undertaken in the area to expand on the project.
\end{itemize}
	

%----------------------------------------------------------------------------------------
%	SECTION 5
%----------------------------------------------------------------------------------------
\section{Research Methodology \& Analytical Approach}
\textit{Suitable techniques will also be used to measure, compare and analyse the performance by means of statistical methods, but also practical considerations related to the complexity of the technique being evaluated.}


%----------------------------------------------------------------------------------------
%	SECTION 6
%----------------------------------------------------------------------------------------
\section{Scope and Limitations}
The project will focus on building an SME credit scorecard, subset of the customers in \subjectname. A holdout dataset will be taken to build the model which will be predicting 12 months in advance.
\\\\
The data used in this project will be taken from the transaction visa debit table, in an ideal world for this analysis you get the location of the actual transactions taking place but this was not available so I had to roll it back to the customers address. Diagram possibly
\\\\ 


%----------------------------------------------------------------------------------------
%	SECTION 7
%----------------------------------------------------------------------------------------
\section{Outline of the Thesis}
The remainder of this thesis is organised as follows:

\begin{itemize}
	\item Chapter 2
	\item Chapter 3
	\item Chapter 4
	\item Chapter 5
	\item Chapter 6
\end{itemize}

